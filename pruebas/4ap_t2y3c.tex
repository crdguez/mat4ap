\documentclass[addpoints,spanish, 12pt,a4paper]{exam}
\pointpoints{punto}{puntos}
\hpword{Puntos:}
\vpword{Puntos:}
\htword{Total}
\vtword{Total}
\hsword{Resultado:}
\hqword{Ejercicio:}
\vqword{Ejercicio:}

\printanswers

\usepackage[utf8]{inputenc}
\usepackage[spanish]{babel}
\usepackage{eurosym}
\usepackage{yhmath}
%\usepackage[spanish,es-lcroman, es-tabla, es-noshorthands]{babel}


\usepackage[margin=1in]{geometry}
\usepackage{amsmath,amssymb}
\usepackage{multicol}

\usepackage{graphicx}
\graphicspath{{../img/}} 

\newcommand{\class}{Matemáticas 4º Aplicadas}
\newcommand{\examdate}{\today}
\newcommand{\examnum}{Números Reales}
\newcommand{\tipo}{A}


\newcommand{\timelimit}{50 minutos}

\renewcommand{\solutiontitle}{\noindent\textbf{Solución:}\enspace}

\pagestyle{head}
\firstpageheader{\includegraphics[width=0.2\columnwidth]{header_left}}{\textbf{Departamento de Matemáticas\linebreak \class}\linebreak \examnum}{\includegraphics[width=0.1\columnwidth]{header_right}}
\runningheader{\class}{\examnum}{Página \thepage\ of \numpages}
\runningheadrule


\begin{document}

\noindent
\begin{tabular*}{\textwidth}{l @{\extracolsep{\fill}} r @{\extracolsep{6pt}} }
\textbf{Nombre:} \makebox[3.5in]{\hrulefill} & \textbf{Fecha:}\makebox[1in]{\hrulefill} \\
 & \\
\textbf{Tiempo: \timelimit} & Tipo: \tipo 
\end{tabular*}
\rule[2ex]{\textwidth}{2pt}
Esta prueba tiene \numquestions\ ejercicios. La puntuación máxima es de \numpoints. 
La nota final de la prueba será la parte proporcional de la puntuación obtenida sobre la puntuación máxima. Para la evaluación de pendientes de 3ºESO o 2ºPMAR se tendrán en cuenta los apartados 1,2,3 y 6: 

\begin{center}


\addpoints
 %\gradetable[h][questions]
	\pointtable[h][questions]
\end{center}

\noindent
\rule[2ex]{\textwidth}{2pt}

\begin{questions}

\question [1] Responde a las siguientes cuestiones:
\begin{parts}
\part Pasa a notación decimal los siguientes números:
\begin{itemize}
\item $ \frac{5}{2}$
\item $ \frac{4}{3}$
\end{itemize}
\begin{solution}\begin{itemize}
\item $ 2,5$
\item $ 1,\wideparen{3}$
\end{itemize} \end{solution}
\part Pasa a fracción irreducible los siguientes números:
\begin{itemize}
\item $ 7,5$
\item $ 6$
\item $ \sqrt{6}$
\end{itemize}
\begin{solution}\begin{itemize}
\item $ \frac{75}{10}=\frac{15}{2}$
\item $ 1,\wideparen{3}$
\item No se puede, es irracional
\end{itemize} \end{solution}
\end{parts}
\addpoints




\question Responde a las siguientes cuestiones:
\begin{parts}
\part[1] Da una aproximación, con tres cifras significativas, para cada una de las
siguientes cantidades:
\begin{itemize}
\item $854 238$ personas
\item $3,1694$ m
\item $928 412$ mg
\end{itemize}
\begin{solution} 
\begin{itemize}
\item 854 miles de personas
\item 3,17 m
\item 928 miles de mg
\end{itemize}
\end{solution}
\part[1] ¿Cuáles son los errores absoluto y relativo cometidos en cada caso?
\begin{solution} \begin{itemize}
\item Error absoluto: 854 238 - 854 000 = 238 personas. Error relativo: $\frac{238}{854238}\approx0,0003 $
\item Error absoluto: 3,1694 - 3,17 = - 0,0006 m. Error relativo: $\frac{-0.0006}{3,1694}\approx-0,0002 $
\item Error absoluto: 412 mg. Error relativo: $\frac{412}{928412}\approx0,0004 $
\end{itemize}
\end{solution}
\end{parts}

\addpoints



\question Responde a las siguientes cuestiones relacionadas con la notación científica:
\begin{parts}
\part[1] Expresa en notación científica cada una de estas cantidades:
\begin{itemize}
\item A = $328 000 000 000$
\item B = $0,000000012$
\end{itemize}
\begin{solution} \begin{itemize}
\item A = $3,28\cdot10^{11}$
\item B = $1,2 \cdot10^{-8}$
\end{itemize}\end{solution}
\part[1] Escribe en forma decimal los siguientes números dados en notación científica:
\begin{itemize}
\item C = $2,25\cdot10^8$
\item D = $3,2\cdot 10^{-4}$
\end{itemize}
\begin{solution} \begin{itemize}
\item C = $225 000 000$
\item D = $0,00032$
\end{itemize}\end{solution}
\part[2] Calcula: $ \left( A + C \right) \cdot B$
\begin{solution}$\left( A + C \right) \cdot B= (3,28 \cdot 10^{11} + 2,25 \cdot 10^8 ) \cdot (1,2 \cdot 10^{-8} ) = (3,28225 \cdot 10^{11} ) \cdot (1,2 \cdot 10^{-8} ) 
= 3,9387 \cdot 10^3 = 3 938,7$ \end{solution}

\end{parts}

\addpoints

\question Responde a las siguientes cuestiones relacionadas con esta operación: $\left(5,28\cdot10^4+2,81\cdot10^5\right)^2$
\begin{parts}
\part[1] Halla el resultado, con ayuda de la calculadora, dando el resultado en notación científica con tres cifras significativas:
\begin{solution} $1,11\cdot10^{11}$
\end{solution}
\part[2] Da una cota para el error absoluto y otra para el error relativo cometidos al dar el valor aproximado.
\begin{solution} $E_a<5\cdot10^8$ y $E_a<\dfrac{5\cdot10^8}{1,11\cdot10^{11}}\approx0,0045$
\end{solution}
\end{parts}

\addpoints



%\question La masa de la Luna es $7,35 \cdot 10^{22}$ kg, la de Mercurio $3,302\cdot10^{23}$ kg y la de la
%Tierra es $5,98\cdot10^{24}$ kg.
%\begin{parts}
%\part[2] Calcula las veces que la masa de la Luna es menor que la masa de Mercurio
%\begin{solution} $3,302 \cdot 10^{23} : 7,35 \cdot 10^{22} = 33,02 \cdot 10^{22} : 7,35 \cdot 10^{22} = 4,49$. La masa de la Luna es 4,49 veces menor que la de Mercurio.
%\end{solution}
%\part[2] Halla la diferencia entre las masas de la Tierra y de Mercurio
%\begin{solution} Diferencia de masa Tierra - Mercurio: $ 5,98 \cdot 10^{24} - 3,302 \cdot 10^{23} = 59,8 \cdot 10^{23} - 3,302 \cdot 10^{23} = 56,498 \cdot 10^{23} = 5,6498 \cdot 10^{24}$ kg
%\end{solution}
%\end{parts}
%
%\addpoints

\question[2] Indica a cuáles de los conjuntos
$\mathbb{N}$, $\mathbb{Z}$, $\mathbb{Q}$, $\mathbb{R}$ pertenecen cada uno de los siguientes números:
\begin{center}
\begin{tabular}{|c |c |c |c |c|}\hline
&$\mathbb{N}$& $\mathbb{Z}$& $\mathbb{Q}$&$\mathbb{R}$\\ 
\hline
$\frac{3}{4}$&&&&\\
\hline
$\sqrt[3]{-27}$&&&&\\
\hline
$1,\wideparen{3}$&&&&\\
\hline
$-\frac{16}{4}$&&&&\\
\hline
$-\sqrt{25}$&&&&\\
\hline
$\sqrt{8}$&&&&\\
\hline
$4$&&&&\\
\hline
$\pi$&&&&\\
\hline
$\sqrt{-4}$&&&&\\
\hline
$\frac{26}{13}$&&&&\\
\hline
\end{tabular}

\end{center}

\begin{solution}
\begin{tabular}{|c |c |c |c |c|}\hline
&$\mathbb{N}$& $\mathbb{Z}$& $\mathbb{Q}$&$\mathbb{R}$\\ 
\hline
$\frac{3}{4}$&&&X&X\\
\hline
$\sqrt[3]{-27}$&&X&X&X\\
\hline
$1,\wideparen{3}$&&&X&X\\
\hline
$-\frac{16}{4}$&&X&X&X\\
\hline
$-\sqrt{25}$&&X&X&X\\
\hline
$\sqrt{8}$&&&&X\\
\hline
$4$&X&X&X&X\\
\hline
$\pi$&&&&X\\
\hline
$\sqrt{-4}$&&&&\\
\hline
$\frac{26}{13}$&X&X&X&X\\
\hline
\end{tabular}
\end{solution}

\addpoints

\question[2] Representa en la recta real y en forma de intervalo el siguiente conjunto numérico:
\addpoints % to omit double points count
$$\left\{ x \in \mathbb{R} \left| -2 \leqslant x < 4 \right. \right\}$$

\begin{solution}
$\left[-2 \ , 4\right)$ 
\end{solution}

\question Calcula los siguientes radicales utilizando la definición o la notación en forma de potencia. (Justifica tus respuestas):
\begin{parts}
\part[1] $\sqrt[3]{2744}$
\begin{solution} $14$ porque $14^3=2744$
\end{solution}
\part[2] $\sqrt{2}\cdot\sqrt{50}$
\begin{solution} $=2^\frac{1}{2}\cdot50^\frac{1}{2}=\left(2\cdot50\right)^\frac{1}{2}=100^\frac{1}{2}=\left(10^2\right)^\frac{1}{2}=10 $
\end{solution}
\end{parts}

\addpoints


\end{questions}

\end{document}
