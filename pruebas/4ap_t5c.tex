\documentclass[addpoints,spanish, 12pt,a4paper]{exam}
\pointpoints{punto}{puntos}
\hpword{Puntos:}
\vpword{Puntos:}
\htword{Total}
\vtword{Total}
\hsword{Resultado:}
\hqword{Ejercicio:}
\vqword{Ejercicio:}

%\printanswers

\usepackage[utf8]{inputenc}
\usepackage[spanish]{babel}
\usepackage{eurosym}
\usepackage{yhmath}
%\usepackage[spanish,es-lcroman, es-tabla, es-noshorthands]{babel}

\usepackage{verbatim}

\usepackage[margin=1in]{geometry}
\usepackage{amsmath,amssymb}
\usepackage{multicol}

\usepackage{graphicx}
\graphicspath{{../img/}} 

\newcommand{\class}{Matemáticas 4º Aplicadas}
\newcommand{\examdate}{\today}
\newcommand{\examnum}{Expresiones algebraicas}
\newcommand{\tipo}{A}


\newcommand{\timelimit}{50 minutos}

\renewcommand{\solutiontitle}{\noindent\textbf{Solución:}\enspace}

\pagestyle{head}
\firstpageheader{\includegraphics[width=0.2\columnwidth]{header_left}}{\textbf{Departamento de Matemáticas\linebreak \class}\linebreak \examnum}{\includegraphics[width=0.1\columnwidth]{header_right}}
\runningheader{\class}{\examnum}{Página \thepage\ of \numpages}
\runningheadrule

\pointsinrightmargin % Para poner las puntuaciones a la derecha. Se puede cambiar. Si se comenta, sale a la izquierda.
\extrawidth{-2.4cm} %Un poquito más de margen por si ponemos textos largos.
\marginpointname{ \emph{\points}}


\begin{document}

\noindent
\begin{tabular*}{\textwidth}{l @{\extracolsep{\fill}} r @{\extracolsep{6pt}} }
\textbf{Nombre:} \makebox[3.5in]{\hrulefill} & \textbf{Fecha:}\makebox[1in]{\hrulefill} \\
 & \\
\textbf{Tiempo: \timelimit} & Tipo: \tipo 
\end{tabular*}
\rule[2ex]{\textwidth}{2pt}
Esta prueba tiene \numquestions\ ejercicios. La puntuación máxima es de \numpoints. 
La nota final de la prueba será la parte proporcional de la puntuación obtenida sobre la puntuación máxima. Para la evaluación de pendientes de 3ºESO o 2ºPMAR se tendrán en cuenta los apartados 1, 2, 4 y 6: 

\begin{center}


\addpoints
 %\gradetable[h][questions]
	\pointtable[h][questions]
\end{center}

\noindent
\rule[2ex]{\textwidth}{2pt}

\begin{questions}

\question Indica el coeficiente, la parte literal y el grado de los siguientes monomios.
 
\begin{parts}
\part [1] $A=6x^3$

\begin{solution}$ $\end{solution}
\part [1] $B=-3x$

\begin{solution}$ $\end{solution}
\part [1] $C=4x^3$

\begin{solution}$ $\end{solution}
\end{parts}

\addpoints


\question [2] Opera y simplifica la siguiente expresión: $$\left(2x^2-3x+1\right)\left(2x-1\right)-\left(4x^3-8x^2+1\right) $$
\begin{solution}
$5x-2$
\end{solution}

\question [2] Multiplica por 6 esta expresión y simplifica: $$\dfrac{2x^2-1}{2}-\dfrac{x-1}{3}-\dfrac{1-x}{6}$$
\begin{solution}
$ 6x^2-x-2$
\end{solution}

\question  Resuelve las siguientes ecuaciones:
\begin{parts}
\part [1] $3(2x+1)-\dfrac{x+1}{2}=\dfrac{1}{2}(x+2-\dfrac{x+1}{3}) $

\begin{solution}$\frac{-10}{31} $\end{solution}
\part [1] $\dfrac{2x+1}{5}-\dfrac{x+1}{3}=\dfrac{3x}{5}-2(\dfrac{x}{6}-1)$

\begin{solution} $\frac{-32}{3}$\end{solution}



\end{parts}
\addpoints

\question Expresa algebraicamente y simplifica los siguientes enunciados:
\begin{parts}
\part [1] La suma de dos números sabiendo que uno es 14 cm mayor que el otro

\begin{solution} $ x+x+14 = 2x+14$ \end{solution}
\part [1] El aŕea de un rectángulo cuya base mide 2 cm más que la altura

\begin{solution} $(x+2)\cdot x=x^2+2x$ \end{solution}


\end{parts}


\question[2] Dos números suman 25 y el doble de uno de ellos es 14. ¿Qué números son? 
\begin{solution} los números son 7 y 18\end{solution}

\question[2] Ana tiene el triple de edad que su hijo Jaime. Dentro de 15 años, la edad de Ana será el doble que la de su hijo. ¿Cuántos años más que Jaime tiene su madre?
\begin{solution} Ana tiene 45 años y su hijo Jaime 15, por tanto, Ana tiene 30 años más que su hijo\end{solution}






\addpoints

\end{questions}

\end{document}
