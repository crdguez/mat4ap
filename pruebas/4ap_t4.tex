\documentclass[addpoints,spanish, 12pt,a4paper]{exam}
\pointpoints{punto}{puntos}
\hpword{Puntos:}
\vpword{Puntos:}
\htword{Total}
\vtword{Total}
\hsword{Resultado:}
\hqword{Ejercicio:}
\vqword{Ejercicio:}

\printanswers

\usepackage[utf8]{inputenc}
\usepackage[spanish]{babel}
\usepackage{eurosym}
\usepackage{yhmath}
%\usepackage[spanish,es-lcroman, es-tabla, es-noshorthands]{babel}

\usepackage{verbatim}

\usepackage[margin=1in]{geometry}
\usepackage{amsmath,amssymb}
\usepackage{multicol}

\usepackage{graphicx}
\graphicspath{{../img/}} 

\newcommand{\class}{Matemáticas 4º Aplicadas}
\newcommand{\examdate}{\today}
\newcommand{\examnum}{Problemas Aritméticos}
\newcommand{\tipo}{A}


\newcommand{\timelimit}{50 minutos}

\renewcommand{\solutiontitle}{\noindent\textbf{Solución:}\enspace}

\pagestyle{head}
\firstpageheader{\includegraphics[width=0.2\columnwidth]{header_left}}{\textbf{Departamento de Matemáticas\linebreak \class}\linebreak \examnum}{\includegraphics[width=0.1\columnwidth]{header_right}}
\runningheader{\class}{\examnum}{Página \thepage\ of \numpages}
\runningheadrule

\pointsinrightmargin % Para poner las puntuaciones a la derecha. Se puede cambiar. Si se comenta, sale a la izquierda.
\extrawidth{-2.4cm} %Un poquito más de margen por si ponemos textos largos.
\marginpointname{ \emph{\points}}


\begin{document}

\noindent
\begin{tabular*}{\textwidth}{l @{\extracolsep{\fill}} r @{\extracolsep{6pt}} }
\textbf{Nombre:} \makebox[3.5in]{\hrulefill} & \textbf{Fecha:}\makebox[1in]{\hrulefill} \\
 & \\
\textbf{Tiempo: \timelimit} & Tipo: \tipo 
\end{tabular*}
\rule[2ex]{\textwidth}{2pt}
Esta prueba tiene \numquestions\ ejercicios. La puntuación máxima es de \numpoints. 
La nota final de la prueba será la parte proporcional de la puntuación obtenida sobre la puntuación máxima. Para la evaluación de pendientes de 3ºESO o 2ºPMAR se tendrán en cuenta los apartados 1,2 y 3: 

\begin{center}


\addpoints
 %\gradetable[h][questions]
	\pointtable[h][questions]
\end{center}

\noindent
\rule[2ex]{\textwidth}{2pt}

\begin{questions}

\question Obtén el valor de x en cada caso:
\begin{parts}
\part [1] x = 2\% de 135

\begin{solution}$135\cdot0,02=2,7$\end{solution}
\part [1] 28\% de x = 15,96

\begin{solution}15,96/0,28=5,7\end{solution}
\end{parts}
\addpoints


\question [2] El precio de un artículo sin IVA es de 315\euro. Si he pagado 365,40\euro, ¿qué porcentaje de IVA me han cargado?
\begin{solution}
Se ha pagado de IVA 365,40 - 315 = 50,40 \euro. 50,40/315 = 0,16. 16\%
\end{solution}

\question [2] He pagado 35,7\euro por una camisa que tenía un 15\% de rebaja. ¿Cuál era su precio antes de estar rebajada?
\begin{solution}
85\% de x = 35,7. 35,7/0,85=42\euro
\end{solution}

\question [2] Carlos coloca 18000\euro \   al 3,5\% anual y los mantiene en el banco durante 3 años, eligiendo la modalidad de interés simple. ¿A cuánto ascienden el capital obtenido durante los tres años?
\begin{solution}
18000x0,03=630. 630x3=1890. 19890\euro
\end{solution}


\question [2] Calcula en cuánto se transforman 9500\euro \ colocados al 3,5\% de interés compuesto anual durante 3 años.
\begin{solution}
$9500\cdot1,035^3=10532,82$\euro
\end{solution}

\question [2] Raquel, María e Isabel han ganado un premio de 8000\euro en un sorteo. Sabiendo que, para comprar los boletos, Raquel puso 5\euro, María 8\euro e Isabel 12\euro, ¿cuánto le corresponderá a cada una del premio que han ganado?
\begin{solution}
1600, 2560, 3840\euro
\end{solution}

\question [3] Una moto sale desde una ciudad A a una velocidad de 44 km/h. Al cabo de media hora, sale un coche desde A que tarda 20 minutos en alcanzarlo. ¿A qué velocidad iba el coche?
\begin{solution}
110 km/h
\end{solution}


\addpoints

\end{questions}

\end{document}
