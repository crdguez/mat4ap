\documentclass[addpoints,spanish, 12pt,a4paper]{exam}
\pointpoints{punto}{puntos}
\hpword{Puntos:}
\vpword{Puntos:}
\htword{Total}
\vtword{Total}
\hsword{Resultado:}
\hqword{Ejercicio:}
\vqword{Ejercicio:}

%\printanswers

\usepackage[utf8]{inputenc}
\usepackage[spanish]{babel}
\usepackage{eurosym}
\usepackage{yhmath}
%\usepackage[spanish,es-lcroman, es-tabla, es-noshorthands]{babel}

\usepackage{verbatim}

\usepackage[margin=1in]{geometry}
\usepackage{amsmath,amssymb}
\usepackage{multicol}

\usepackage{graphicx}
\graphicspath{{../img/}} 

\newcommand{\class}{Matemáticas 4º Aplicadas}
\newcommand{\examdate}{\today}
\newcommand{\examnum}{Ecuaciones y Sistemas}
\newcommand{\tipo}{A}


\newcommand{\timelimit}{50 minutos}

\renewcommand{\solutiontitle}{\noindent\textbf{Solución:}\enspace}

\pagestyle{head}
\firstpageheader{\includegraphics[width=0.2\columnwidth]{header_left}}{\textbf{Departamento de Matemáticas\linebreak \class}\linebreak \examnum}{\includegraphics[width=0.1\columnwidth]{header_right}}
\runningheader{\class}{\examnum}{Página \thepage\ of \numpages}
\runningheadrule

\pointsinrightmargin % Para poner las puntuaciones a la derecha. Se puede cambiar. Si se comenta, sale a la izquierda.
\extrawidth{-2.4cm} %Un poquito más de margen por si ponemos textos largos.
\marginpointname{ \emph{\points}}


\begin{document}

\noindent
\begin{tabular*}{\textwidth}{l @{\extracolsep{\fill}} r @{\extracolsep{6pt}} }
\textbf{Nombre:} \makebox[3.5in]{\hrulefill} & \textbf{Fecha:}\makebox[1in]{\hrulefill} \\
 & \\
\textbf{Tiempo: \timelimit} & Tipo: \tipo 
\end{tabular*}
\rule[2ex]{\textwidth}{2pt}
Esta prueba tiene \numquestions\ ejercicios. La puntuación máxima es de \numpoints. 
La nota final de la prueba será la parte proporcional de la puntuación obtenida sobre la puntuación máxima. Para la evaluación de pendientes de 3ºESO o 2ºPMAR se tendrán en cuenta los apartados 1.a, 1.c, 1.d, 2.a y 4: 

\begin{center}


\addpoints
 %\gradetable[h][questions]
	\pointtable[h][questions]
\end{center}

\noindent
\textbf{NOTA:} Los problemas se han de resolver mediante ecuaciones o sistemas. Y los ejercicios mediante métodos diferentes a la resolución por tanteo.
\rule[2ex]{\textwidth}{2pt}

\begin{questions}

\question Resuelve las siguientes ecuaciones
\begin{parts}
%solve(Eq(2*(x-3)-5*x+7,11*(1-x)-(1+3*x)-x))
\part[1] $2(x-3)-5x+7=11(1-x)-(1+3x)-x$
\begin{solution}
$x=\frac{3}{4}$
\end{solution}
\part[1] $x+\dfrac{3(x-2)}{9}=\dfrac{5(x-1)}{4}+\dfrac{7}{12}$
\begin{solution}
$x=0$
\end{solution}
\part[1] $x^2-2x-8=0$  
\begin{solution} $x_1=4$, $x_2=-2$ 
\end{solution}
\part[2] $15-(x+2)^2=(x-3)^2+2x$
\begin{solution}
$x=1 \land x=-1$
\end{solution}
\end{parts}

\question Resolver los sistemas de ecuaciones que siguen:
\begin{parts}
\part[1] $\left. \begin{gathered}
	  4x - 2y = 16 \hfill \\
	  3x - 7y = 1 \hfill \\ 
	\end{gathered}  \right\}$
\begin{solution} x=5; y=2 \end{solution}

\part[2] $\left. \begin{gathered}
	  \frac{x}{2} - \frac{y}{3} = 2 \hfill \\
	  \frac{{x - 1}}{3} + \frac{{y - 2}}{2} = \frac{{13}}{6} \hfill \\ 
	\end{gathered}  \right\}$
\begin{solution} x=6; y=3 \end{solution}
\end{parts}

\question[1] Cuatro barras de pan y seis litros de leche cuestan 6,80 ; tres barras de pan y cuatro
litros de leche cuestan 4,70. ¿Cuánto vale una barra de pan? ¿Cuánto cuesta un
litro de leche?
\begin{solution} $\mathrm{~} \begin{cases} 4 x + 6 y = 680\\3 x + 4 y = 470\end{cases} \to \begin{pmatrix}50, & 80\end{pmatrix}$\end{solution}

\question[1] Encuentra dos números tales que añadiendo tres unidades al primero se obtenga el
segundo y, en cambio, añadiendo dos unidades al segundo se obtenga el doble del primero.
\begin{solution} $\mathrm{~} \begin{cases} x + 3 = y\\y + 2 = 2 x\end{cases} \to \begin{pmatrix}5, & 8\end{pmatrix}$\end{solution}

\question[2] Si a un número de dos cifras le sumamos 18 se obtiene un número con las cifras intercambiadas entre sí. Sabiendo que la suma de las cifras de ese número es 16, encuéntralo.
\begin{solution} Número 79 \end{solution}

\addpoints

\end{questions}

\end{document}
