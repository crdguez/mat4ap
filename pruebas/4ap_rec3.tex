\documentclass[addpoints,spanish, 12pt,a4paper]{exam}
\pointpoints{punto}{puntos}
\hpword{Puntos:}
\vpword{Puntos:}
\htword{Total}
\vtword{Total}
\hsword{Resultado:}
\hqword{Ejercicio:}
\vqword{Ejercicio:}
\usepackage{pgf,tikz}
\usetikzlibrary{shapes, calc, shapes, arrows, math, babel}

\printanswers

\usepackage[utf8]{inputenc}
\usepackage[spanish]{babel}
\usepackage{eurosym}
\usepackage{yhmath}
%\usepackage[spanish,es-lcroman, es-tabla, es-noshorthands]{babel}

\usepackage{verbatim}

\usepackage[margin=1in]{geometry}
\usepackage{amsmath,amssymb}
\usepackage{multicol}

\usepackage{graphicx}
\graphicspath{{../img/}} 

\newcommand{\class}{Matemáticas 4º Aplicadas}
\newcommand{\examdate}{\today}
\newcommand{\examnum}{Recuperación 3ªEv.}
\newcommand{\tipo}{A}


\newcommand{\timelimit}{50 minutos}

\renewcommand{\solutiontitle}{\noindent\textbf{Solución:}\enspace}

\pagestyle{head}
\firstpageheader{\includegraphics[width=0.2\columnwidth]{header_left}}{\textbf{Departamento de Matemáticas\linebreak \class}\linebreak \examnum}{\includegraphics[width=0.1\columnwidth]{header_right}}
\runningheader{\class}{\examnum}{Página \thepage\ of \numpages}
\runningheadrule

\pointsinrightmargin % Para poner las puntuaciones a la derecha. Se puede cambiar. Si se comenta, sale a la izquierda.
\extrawidth{-2.4cm} %Un poquito más de margen por si ponemos textos largos.
\marginpointname{ \emph{\points}}


\begin{document}

\noindent
\begin{tabular*}{\textwidth}{l @{\extracolsep{\fill}} r @{\extracolsep{6pt}} }
\textbf{Nombre:} \makebox[3.5in]{\hrulefill} & \textbf{Fecha:}\makebox[1in]{\hrulefill} \\
 & \\
\textbf{Tiempo: \timelimit} & Tipo: \tipo 
\end{tabular*}
\rule[2ex]{\textwidth}{2pt}
Esta prueba tiene \numquestions\ ejercicios. La puntuación máxima es de \numpoints. 
La nota final de la prueba será la parte proporcional de la puntuación obtenida sobre la puntuación máxima. 

\begin{center}


\addpoints
 %\gradetable[h][questions]
	\pointtable[h][questions]
\end{center}

\noindent
\textbf{NOTA:} Los problemas se han de resolver mediante ecuaciones o sistemas. Y los ejercicios mediante métodos diferentes a la resolución por tanteo.
\rule[2ex]{\textwidth}{2pt}

\begin{questions}

\question Resuelve las siguientes ecuaciones
\begin{parts}
%solve(Eq(2*(x-3)-5*x+7,11*(1-x)-(1+3*x)-x))
\part[1] $2(x-3)-5x+7=11(1-x)-(1+3x)-x$
\begin{solution}
$x=\frac{3}{4}$
\end{solution}
\part[1] $x+\dfrac{3(x-2)}{9}=\dfrac{5(x-1)}{4}+\dfrac{7}{12}$
\begin{solution}
$x=0$
\end{solution}
\part[1] $x^2-2x-8=0$  
\begin{solution} $x_1=4$, $x_2=-2$ 
\end{solution}
\begin{comment}
\part[2] $15-(x+2)^2=(x-3)^2+2x$
\begin{solution}
$x=1 \land x=-1$
\end{solution}
\end{comment}
\end{parts}

\question Resolver los sistemas de ecuaciones que siguen:
\begin{parts}
\part[1] $\left. \begin{gathered}
	  4x - 2y = 16 \hfill \\
	  3x - 7y = 1 \hfill \\ 
	\end{gathered}  \right\}$
\begin{solution} x=5; y=2 \end{solution}
\begin{comment}
\part[2] $\left. \begin{gathered}
	  \frac{x}{2} - \frac{y}{3} = 2 \hfill \\
	  \frac{{x - 1}}{3} + \frac{{y - 2}}{2} = \frac{{13}}{6} \hfill \\ 
	\end{gathered}  \right\}$
\begin{solution} x=6; y=3 \end{solution}
\end{comment}
\end{parts}

\question[1] Cuatro barras de pan y seis litros de leche cuestan 6,80 ; tres barras de pan y cuatro
litros de leche cuestan 4,70. ¿Cuánto vale una barra de pan? ¿Cuánto cuesta un
litro de leche?
\begin{solution} $\mathrm{~} \begin{cases} 4 x + 6 y = 680\\3 x + 4 y = 470\end{cases} \to \begin{pmatrix}50, & 80\end{pmatrix}$\end{solution}

\question Una compañía de teléfonos me cobra una cantidad fija al mes: 2 \euro . Además me cobran 50 centimos por cada hora de llamadas. Queremos reflejar en forma de función la factura mensual (lo que pago al mes)
\begin{parts}
\part[1] ¿Cuáles son la variables dependientes e independientes de la función? Haz una tabla de valores que refleje dicha variable
\begin{solution}
$x$ = tiempo, $y$ = dinero \\
\begin{center}
\begin{tabular}{|c |c |}\hline
$x$ & $y$\\ 
\hline
$1$&$2.5$\\
\hline
$2$&$3$\\
\hline
$3$&$3.5$\\
\hline
$4$&$4$\\
\hline
\end{tabular}
\end{center}
\end{solution}

\part[1] Representa gráficamente los valores anteriores y únelos para determinar la gráfica de la función\\
\begin{tikzpicture}[line cap=round,line join=round,>=triangle 45,x=1cm,y=1cm, scale=0.6]
\draw [color=lightgray,dash pattern=on 1pt off 1pt, xstep=1cm,ystep=1cm] (-3.6,-3.4) grid (20.1,10.1);
\draw[<->,color=black] (-3.6,0) -- (20.1,0);
\foreach \x in {-3,-2,-1,1,2,3,4,5,6,7,8,9,10,11,12,13,14,15,16,17,18,19,20}
\draw[shift={(\x,0)},color=black] (0pt,1pt) -- (0pt,-1pt) node[below] {\footnotesize $\x$};
\draw[<->,color=black] (0,-3.43158220601634095) -- (0,10.1);
\foreach \y in {-3,-2,-1,1,2,3,4,5,6,7,8,9,10}
\draw[shift={(0,\y)},color=black] (2pt,0pt) -- (-2pt,0pt) node[left] {\footnotesize $\y$};
%\draw[color=black] (0pt,-10pt) node[right] {\footnotesize $0$};
%\clip(-0.6129302567150502,-0.43158220601634095) rectangle (9.010648940148005,7.8783927087822985);
\end{tikzpicture}
\begin{solution}
\begin{tikzpicture}[line cap=round,line join=round,>=triangle 45,x=1cm,y=1cm, scale=0.4]
\draw [color=lightgray,dash pattern=on 1pt off 1pt, xstep=1cm,ystep=1cm] (-3.6,-3.4) grid (20.1,10.1);
\draw[<->,color=black] (-3.6,0) -- (20.1,0);
\foreach \x in {-3,-2,-1,1,2,3,4,5,6,7,8,9,10,11,12,13,14,15,16,17,18,19,20}
\draw[shift={(\x,0)},color=black] (0pt,1pt) -- (0pt,-1pt) node[below] {\footnotesize $\x$};
\draw[<->,color=black] (0,-3.43158220601634095) -- (0,10.1);
\foreach \y in {-3,-2,-1,1,2,3,4,5,6,7,8,9,10}
\draw[shift={(0,\y)},color=black] (2pt,0pt) -- (-2pt,0pt) node[left] {\footnotesize $\y$};
%\draw[color=black] (0pt,-10pt) node[right] {\footnotesize $0$};
%\clip(-0.6129302567150502,-0.43158220601634095) rectangle (9.010648940148005,7.8783927087822985);
\draw[->, color=red, domain=0  :16 + 0.1]    plot (\x,{1/2*(\x) + 2}) node[right] {};
\end{tikzpicture}

\end{solution}
\part[1] Da la expresión analítica (o algebraica) de la función. Con dicha expresión calcula lo que me facturarían un mes que hablara 30 horas
\begin{solution} $y=0.5x+2$ e $y=0.5\cdot30 +2=17$
\end{solution}
\end{parts}


\question[1] Hemos salido a medir el edificio. Y hemos obtenido los siguientes datos.\begin{itemize}
\item La sombra del edificio es de 15.6 metros
\item La altura de otra persona es 1.80 mts y su sombra es 2.34 mts
\end{itemize}
Determina la altura del edifico
\begin{solution} $\dfrac{x}{1560}=\dfrac{180}{234} \to x=\dfrac{180\cdot1560}{234}= 1200 cm$  \end{solution}

\question Tenemos un tupperware (recipiente de plástico con forma de prisma para guardar alimentos) de dimensiones: 20cm de largo, 10cm de ancho y 8cm de alto:
\begin{parts}
\part[1] Si queremos pintar todas sus caras exteriores, ¿cuántos botes de pintura necesitaré si con un bote pinto un metro cuadrado de superficie?
\begin{solution} (200, 80, 160, 880, 0.088) \end{solution}

\part[1] ¿Cuánta agua cabría  sabiendo que un litro es lo mismo que un decímetro cúbico?
\begin{solution} (1600.0, 1.6)  \end{solution}
\end{parts}


\addpoints

\end{questions}

\end{document}
