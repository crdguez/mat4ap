\documentclass[addpoints,spanish, 12pt,a4paper]{exam}
\pointpoints{punto}{puntos}
\hpword{Puntos:}
\vpword{Puntos:}
\htword{Total}
\vtword{Total}
\hsword{Resultado:}
\hqword{Ejercicio:}
\vqword{Ejercicio:}
\usepackage{pgf,tikz}
\usetikzlibrary{shapes, calc, shapes, arrows, math, babel}

\printanswers

\usepackage[utf8]{inputenc}
\usepackage[spanish]{babel}
\usepackage{eurosym}
\usepackage{yhmath}
%\usepackage[spanish,es-lcroman, es-tabla, es-noshorthands]{babel}

\usepackage{verbatim}

\usepackage[margin=1in]{geometry}
\usepackage{amsmath,amssymb}
\usepackage{multicol}

\usepackage{graphicx}
\graphicspath{{../img/}} 

\newcommand{\class}{Matemáticas 4º Aplicadas}
\newcommand{\examdate}{\today}
\newcommand{\examnum}{Global}
\newcommand{\tipo}{A}


\newcommand{\timelimit}{50 minutos}

\renewcommand{\solutiontitle}{\noindent\textbf{Solución:}\enspace}

\pagestyle{head}
\firstpageheader{\includegraphics[width=0.2\columnwidth]{header_left}}{\textbf{Departamento de Matemáticas\linebreak \class}\linebreak \examnum}{\includegraphics[width=0.1\columnwidth]{header_right}}
\runningheader{\class}{\examnum}{Página \thepage\ of \numpages}
\runningheadrule

\pointsinrightmargin % Para poner las puntuaciones a la derecha. Se puede cambiar. Si se comenta, sale a la izquierda.
\extrawidth{-2.4cm} %Un poquito más de margen por si ponemos textos largos.
\marginpointname{ \emph{\points}}


\begin{document}

\noindent
\begin{tabular*}{\textwidth}{l @{\extracolsep{\fill}} r @{\extracolsep{6pt}} }
\textbf{Nombre:} \makebox[3.5in]{\hrulefill} & \textbf{Fecha:}\makebox[1in]{\hrulefill} \\
 & \\
\textbf{Tiempo: \timelimit} & Tipo: \tipo 
\end{tabular*}
\rule[2ex]{\textwidth}{2pt}
\begin{comment}
Esta prueba tiene \numquestions\ ejercicios. La puntuación máxima es de \numpoints. 
La nota final de la prueba será la parte proporcional de la puntuación obtenida sobre la puntuación máxima. 

\begin{center}


\addpoints
 %\gradetable[h][questions]
	\pointtable[h][questions]
\end{center}

\noindent
\end{comment}
\textbf{Instrucciones:} \begin{itemize}
\item \textbf{Si tienes alguna/s evaluación pendiente:} Tienes que hacer \textbf{todos} los ejercicios salvo el último
\item \textbf{Si tienes todas las evaluaciones aprobadas:} Tienes que hacer el \textbf{último ejercicio}, y luego del resto cuatro ejercicios
\end{itemize}
\rule[2ex]{\textwidth}{2pt}

\begin{questions}

\question Efectúa y simplifica:
\noaddpoints % to omit double points count
\begin{parts}
\part[1] $\dfrac{3}{2}-\dfrac{4}{5}:\dfrac{1}{2}+\dfrac{3}{4}\cdot\dfrac{1}{3}$ 
\begin{solution}$= \dfrac{3}{2}-\dfrac{8}{5}+\dfrac{1}{4}=\dfrac{30}{20}-\dfrac{32}{20}+\dfrac{5}{20}=\dfrac{3}{20}$ \end{solution}
\begin{comment}
\part[1] $\dfrac{1}{6}-\dfrac{5}{3}\left(\dfrac{4}{5}-\dfrac{1}{3}\right)-\dfrac{1}{2}:\dfrac{3}{4}$
\begin{solution}$=\dfrac{1}{6}-\dfrac{5}{3}\cdot\dfrac{7}{15}-\dfrac{2}{3}=\dfrac{1}{6}-\dfrac{7}{9}-\dfrac{2}{3}=\dfrac{3}{18}-\dfrac{14}{18}-\dfrac{12}{18}=-\dfrac{23}{18}$ \end{solution}
\end{comment}
\part[1] $\left(\dfrac{3}{5}-\dfrac{2}{3}\right)^3\cdot\left[\left(\dfrac{1}{2}+\dfrac{1}{3}\right)^3:\left(\dfrac{5}{3}-1\right)^3\right]$
\begin{solution}$=\left(\dfrac{9-10}{15}\right)^3\cdot\left[\left(\dfrac{3+2}{6}\right)^3:\left(\dfrac{2}{3}\right)^3\right]=$ \\ 
$=\left(-\dfrac{1}{15}\right)^3\cdot\left[\left(\dfrac{5}{6}\right)^3:\left(\dfrac{2}{3}\right)^3\right]=\left(-\dfrac{1}{15}\right)^3\cdot\left(\dfrac{5}{4}\right)^3=\left(-\dfrac{1}{12}\right)^3=-\dfrac{1}{1728}$ \end{solution}
\end{parts}

\question[1] Pablo gasta 2/5 del dinero que tenía en comprar fruta. Después, gasta 1/4 de lo que le queda en comprar leche. Sabiendo que le han sobrado 9 \euro. ¿Cuánto dinero tenía al principio?
\begin{solution} Fracción gastada: $\frac{2}{5}+\frac{1}{4}\cdot\frac{3}{5}=\frac{8}{20}+\frac{3}{20}=\frac{11}{20}$  \\
Fracción que le queda: $1-\frac{11}{20}=\frac{9}{20}$ \\
Dinero que tenía: Si $\frac{9}{20}$ del Total $= 9\to$ Total $=\dfrac{9\cdot20}{9}=20$ \euro. \end{solution}

\question [2] Raquel, María e Isabel han ganado un premio de 8000\euro en un sorteo. Sabiendo que, para comprar los boletos, Raquel puso 5\euro, María 8\euro e Isabel 12\euro, ¿cuánto le corresponderá a cada una del premio que han ganado?
\begin{solution}
1600, 2560, 3840\euro
\end{solution}


\question Resuelve las siguientes ecuaciones
\begin{parts}
%solve(Eq(2*(x-3)-5*x+7,11*(1-x)-(1+3*x)-x))
\part[1] $2(x-3)-5x+7=11(1-x)-(1+3x)-x$
\begin{solution}
$x=\frac{3}{4}$
\end{solution}
\part[1] $x+\dfrac{3(x-2)}{9}=\dfrac{5(x-1)}{4}+\dfrac{7}{12}$
\begin{solution}
$x=0$
\end{solution}
\part[1] $x^2-2x-8=0$  
\begin{solution} $x_1=4$, $x_2=-2$ 
\end{solution}
\begin{comment}
\part[2] $15-(x+2)^2=(x-3)^2+2x$
\begin{solution}
$x=1 \land x=-1$
\end{solution}
\end{comment}
\end{parts}

\question Resolver los sistemas de ecuaciones que siguen:
\begin{parts}
\part[1] $\left. \begin{gathered}
	  4x - 2y = 16 \hfill \\
	  3x - 7y = 1 \hfill \\ 
	\end{gathered}  \right\}$
\begin{solution} x=5; y=2 \end{solution}
\begin{comment}
\part[2] $\left. \begin{gathered}
	  \frac{x}{2} - \frac{y}{3} = 2 \hfill \\
	  \frac{{x - 1}}{3} + \frac{{y - 2}}{2} = \frac{{13}}{6} \hfill \\ 
	\end{gathered}  \right\}$
\begin{solution} x=6; y=3 \end{solution}
\end{comment}
\end{parts}

\question[1] Cuatro barras de pan y seis litros de leche cuestan 6,80 ; tres barras de pan y cuatro
litros de leche cuestan 4,70. ¿Cuánto vale una barra de pan? ¿Cuánto cuesta un
litro de leche?
\begin{solution} $\mathrm{~} \begin{cases} 4 x + 6 y = 680\\3 x + 4 y = 470\end{cases} \to \begin{pmatrix}50, & 80\end{pmatrix}$\end{solution}

\begin {comment}
\question[1] Javier tiene 27 años más que su hija Nuria. Dentro 
de ocho años, la edad de Javier doblará la de Nuria.¿Cuántos años tiene cada uno? 
\begin{solution}
Javier, 46 años, y Nuria, 19)
\end{solution}
\end{comment}

\question Una compañía de teléfonos me cobra una cantidad fija al mes: 3.5 \euro . Además me cobran 25 centimos por cada hora de llamadas. Queremos reflejar en forma de función la factura mensual (lo que pago al mes)
\begin{parts}
\part[1] ¿Cuáles son la variables dependientes e independientes de la función? Haz una tabla de valores que refleje dicha variable
\begin{solution}
$x$ = tiempo, $y$ = dinero \\
\begin{center}
\begin{tabular}{|c |c |}\hline
$x$ & $y$\\ 
\hline
$1$&$2.5$\\
\hline
$2$&$3$\\
\hline
$3$&$3.5$\\
\hline
$4$&$4$\\
\hline
\end{tabular}
\end{center}
\end{solution}

\part[1] Representa gráficamente los valores anteriores y únelos para determinar la gráfica de la función\\
\begin{tikzpicture}[line cap=round,line join=round,>=triangle 45,x=1cm,y=1cm, scale=0.6]
\draw [color=lightgray,dash pattern=on 1pt off 1pt, xstep=1cm,ystep=1cm] (-3.6,-3.4) grid (20.1,10.1);
\draw[<->,color=black] (-3.6,0) -- (20.1,0);
\foreach \x in {-3,-2,-1,1,2,3,4,5,6,7,8,9,10,11,12,13,14,15,16,17,18,19,20}
\draw[shift={(\x,0)},color=black] (0pt,1pt) -- (0pt,-1pt) node[below] {\footnotesize $\x$};
\draw[<->,color=black] (0,-3.43158220601634095) -- (0,10.1);
\foreach \y in {-3,-2,-1,1,2,3,4,5,6,7,8,9,10}
\draw[shift={(0,\y)},color=black] (2pt,0pt) -- (-2pt,0pt) node[left] {\footnotesize $\y$};
%\draw[color=black] (0pt,-10pt) node[right] {\footnotesize $0$};
%\clip(-0.6129302567150502,-0.43158220601634095) rectangle (9.010648940148005,7.8783927087822985);
\end{tikzpicture}
\begin{solution}
\begin{tikzpicture}[line cap=round,line join=round,>=triangle 45,x=1cm,y=1cm, scale=0.4]
\draw [color=lightgray,dash pattern=on 1pt off 1pt, xstep=1cm,ystep=1cm] (-3.6,-3.4) grid (20.1,10.1);
\draw[<->,color=black] (-3.6,0) -- (20.1,0);
\foreach \x in {-3,-2,-1,1,2,3,4,5,6,7,8,9,10,11,12,13,14,15,16,17,18,19,20}
\draw[shift={(\x,0)},color=black] (0pt,1pt) -- (0pt,-1pt) node[below] {\footnotesize $\x$};
\draw[<->,color=black] (0,-3.43158220601634095) -- (0,10.1);
\foreach \y in {-3,-2,-1,1,2,3,4,5,6,7,8,9,10}
\draw[shift={(0,\y)},color=black] (2pt,0pt) -- (-2pt,0pt) node[left] {\footnotesize $\y$};
%\draw[color=black] (0pt,-10pt) node[right] {\footnotesize $0$};
%\clip(-0.6129302567150502,-0.43158220601634095) rectangle (9.010648940148005,7.8783927087822985);
\draw[->, color=red, domain=0  :16 + 0.1]    plot (\x,{1/2*(\x) + 2}) node[right] {};
\end{tikzpicture}

\end{solution}
\part[1] Da la expresión analítica (o algebraica) de la función. Con dicha expresión calcula lo que me facturarían un mes que hablara 30 horas
\part[1] Indica el dominio y el recorrido de la función
\begin{solution} $y=0.5x+2$ e $y=0.5\cdot30 +2=17$
\end{solution}
\end{parts}

\question[1] Se tiene una piscina con las siguientes dimensiones:

\pgfdeclarelayer{nodelayer}
\pgfdeclarelayer{edgelayer}
\pgfsetlayers{edgelayer,nodelayer,main}

\tikzset{none/.style={thick}}
\tikzset{simple/.style={thick}}

\begin{tikzpicture}[scale=0.4]
	\begin{pgfonlayer}{nodelayer}
		\node [style=none] (0) at (-0.75, 7) {};
		\node [style=none] (1) at (2, 5) {};
		\node [style=none] (2) at (-0.5, 2.25) {};
		\node [style=none] (3) at (2.2, 0.25) {};
		\node [style=none] (4) at (0.2, -2.225) {};
		\node [style=none] (5) at (-5.25, 2.025) {};
		\node [style=none] (6) at (-5.25, 0.225) {};
		\node [style=none] (7) at (0.2, -3.775) {};
		\node [style=none] (8) at (2.2, -1.3) {};
		\node [style=none] (9) at (2, 3.425) {};
		\node [style=none] (10) at (0.375, 1.575) {};
		\node [style=none] (11) at (-3.25, 5) {};
		\node [style=none] (12) at (-3.25, 5) {};
		\node [style=none] (13) at (-3.25, 5) {20 m};
		\node [style=none] (14) at (1.25, 6.5) {};
		\node [style=none] (15) at (1.25, 6.5) {12 m};
		\node [style=none] (19) at (0.5, 1) {10 m};
		\node [style=none] (20) at (0.75, -0.5) {8 m};
		\node [style=none] (21) at (2.5, 4.25) {};
		\node [style=none] (22) at (2.5, 4.25) {2 m};
	\end{pgfonlayer}
	\begin{pgfonlayer}{edgelayer}
		\draw (5.center) to (0.center);
		\draw (5.center) to (4.center);
		\draw (4.center) to (3.center);
		\draw (2.center) to (1.center);
		\draw (1.center) to (0.center);
		\draw (5.center) to (6.center);
		\draw (6.center) to (7.center);
		\draw (7.center) to (8.center);
		\draw (8.center) to (3.center);
		\draw (7.center) to (4.center);
		\draw (10.center) to (9.center);
		\draw (9.center) to (1.center);
		\draw (2.center) to (10.center);
		\draw (10.center) to (3.center);
	\end{pgfonlayer}
\end{tikzpicture}

\begin{parts}
\part ¿Qué capacidad tiene?¿Cuántos litros caben? 
\begin{solution}$(10\cdot 8 \cdot 2) +(12\cdot20\cdot 2)=160+480=640 m^3=640000 l$ \end{solution}
\part ¿Qué superficie tienen en total entre las paredes y el fondo?
\begin{solution}Lateral: $8\cdot2+10\cdot2+12\cdot2+12\cdot2+20\cdot2+22\cdot2=168m^2$\\
Fondo:
$20\cdot 12 + 8 \cdot 10 =320m^2 $\\
Total = $488m^2$

\end{solution}

\part ¿Cuántos botes de pintura necesitaré para pintarla si con un bote pinto 10 metros cuadrados?
\begin{solution}$488/10=48.8\approx49$ botes\end{solution}
\end{parts}







\addpoints

\end{questions}

\end{document}
