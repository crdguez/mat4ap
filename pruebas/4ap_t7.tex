\documentclass[addpoints,spanish, 12pt,a4paper]{exam}
\pointpoints{punto}{puntos}
\hpword{Puntos:}
\vpword{Puntos:}
\htword{Total}
\vtword{Total}
\hsword{Resultado:}
\hqword{Ejercicio:}
\vqword{Ejercicio:}
\usepackage{pgf,tikz}
\usetikzlibrary{shapes, calc, shapes, arrows, math, babel}

%\printanswers

\usepackage[utf8]{inputenc}
\usepackage[spanish]{babel}
\usepackage{eurosym}
\usepackage{yhmath}
%\usepackage[spanish,es-lcroman, es-tabla, es-noshorthands]{babel}

\usepackage{verbatim}

\usepackage[margin=1in]{geometry}
\usepackage{amsmath,amssymb}
\usepackage{multicol}

\usepackage{graphicx}
\graphicspath{{../img/}} 

\newcommand{\class}{Matemáticas 4º Aplicadas}
\newcommand{\examdate}{\today}
\newcommand{\examnum}{Funciones y Geometría}
\newcommand{\tipo}{A}


\newcommand{\timelimit}{50 minutos}

\renewcommand{\solutiontitle}{\noindent\textbf{Solución:}\enspace}

\pagestyle{head}
\firstpageheader{\includegraphics[width=0.2\columnwidth]{header_left}}{\textbf{Departamento de Matemáticas\linebreak \class}\linebreak \examnum}{\includegraphics[width=0.1\columnwidth]{header_right}}
\runningheader{\class}{\examnum}{Página \thepage\ of \numpages}
\runningheadrule

\pointsinrightmargin % Para poner las puntuaciones a la derecha. Se puede cambiar. Si se comenta, sale a la izquierda.
\extrawidth{-2.4cm} %Un poquito más de margen por si ponemos textos largos.
\marginpointname{ \emph{\points}}


\begin{document}

\noindent
\begin{tabular*}{\textwidth}{l @{\extracolsep{\fill}} r @{\extracolsep{6pt}} }
\textbf{Nombre:} \makebox[3.5in]{\hrulefill} & \textbf{Fecha:}\makebox[1in]{\hrulefill} \\
 & \\
\textbf{Tiempo: \timelimit} & Tipo: \tipo 
\end{tabular*}
\rule[2ex]{\textwidth}{2pt}
Esta prueba tiene \numquestions\ ejercicios. La puntuación máxima es de \numpoints. 
La nota final de la prueba será la parte proporcional de la puntuación obtenida sobre la puntuación máxima. Para la evaluación de pendientes de 3ºESO o 2ºPMAR se tendrán en cuenta los apartados 1.a, 1.c, 1.d, 2.a y 4: 

\begin{center}


\addpoints
 %\gradetable[h][questions]
	\pointtable[h][questions]
\end{center}

\noindent
\textbf{NOTA:} Los problemas se han de resolver mediante ecuaciones o sistemas. Y los ejercicios mediante métodos diferentes a la resolución por tanteo.
\rule[2ex]{\textwidth}{2pt}

\begin{questions}

\question Una compañía de teléfonos me cobra una cantidad fija al mes: 1 \euro . Además me cobran 50 céntimos por cada hora de llamadas. Queremos reflejar en forma de función la factura mensual (lo que pago al mes)
\begin{parts}
\part[1] ¿Cuáles son la variables dependientes e independientes de la función?
\part[1] Haz una tabla de valores que refleje dicha variable
\part[1] Representa gráficamente los valores anteriores y únelos para determinar la gráfica de la función\\
\begin{tikzpicture}[line cap=round,line join=round,>=triangle 45,x=1cm,y=1cm, scale=0.6]
\draw [color=lightgray,dash pattern=on 1pt off 1pt, xstep=1cm,ystep=1cm] (-3.6,-3.4) grid (20.1,10.1);
\draw[<->,color=black] (-3.6,0) -- (20.1,0);
\foreach \x in {-3,-2,-1,1,2,3,4,5,6,7,8,9,10,11,12,13,14,15,16,17,18,19,20}
\draw[shift={(\x,0)},color=black] (0pt,1pt) -- (0pt,-1pt) node[below] {\footnotesize $\x$};
\draw[<->,color=black] (0,-3.43158220601634095) -- (0,10.1);
\foreach \y in {-3,-2,-1,1,2,3,4,5,6,7,8,9,10}
\draw[shift={(0,\y)},color=black] (2pt,0pt) -- (-2pt,0pt) node[left] {\footnotesize $\y$};
%\draw[color=black] (0pt,-10pt) node[right] {\footnotesize $0$};
%\clip(-0.6129302567150502,-0.43158220601634095) rectangle (9.010648940148005,7.8783927087822985);
\end{tikzpicture}
\part[1] Da la expresión analítica (o algebraica) de la función
\part[1] A partir de la expresión analítica, calcula cuánto me facturarán si un mes hablo 200 horas 
\end{parts}

\question[1] Hemos salido a medir el edificio. Y hemos obtenido los siguientes datos.\begin{itemize}
\item La sombra del edificio es de 9.23 metros
\item La altura de una persona es 1.70 mts y su sombra es 2.21 mts
\item La altura de otra persona es 1.80 mts y su sombra es 2.34 mts
\end{itemize}
Determina la altura del edifico
\begin{solution} 12 \end{solution}

\question Tenemos un Tupperware de dimensiones: 20cm de largo, 10cm de ancho y 8cm de alto:
\begin{parts}
\part[1] Si queremos pintarlo, ¿cuánta pintura necesitaré si con un bote pinto un metro cuadrado de superficie?
\part[1] ¿Cuántos litros de sopa cabrán en el tupper sabiendo que un litro es lo mismo que un decímetro cúbico?
\part[1] ¿Cuánto pesará el tupper lleno sabiendo que 1 litro de sopa pesa un kilogramo?
\end{parts}

\question Sabemos que la pirámide de Kefrén tiene 136 mts de altura y el lado de la base 215:
\begin{parts}
\part[2] Si queremos pintarlo, ¿cuánta pintura necesitaré si con un bote pinto un metro cuadrado de superficie?
\part[1] ¿Cuántos litros de sopa cabrían en la pirámide si fuera hueca que un litro es lo mismo que un decímetro cúbico?
\end{parts}


\addpoints

\end{questions}

\end{document}
