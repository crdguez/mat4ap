\documentclass[addpoints,spanish, 12pt,a4paper]{exam}
\pointpoints{punto}{puntos}
\hpword{Puntos:}
\vpword{Puntos:}
\htword{Total}
\vtword{Total}
\hsword{Resultado:}
\hqword{Ejercicio:}
\vqword{Ejercicio:}

\printanswers

\usepackage[utf8]{inputenc}
\usepackage[spanish]{babel}
\usepackage{eurosym}
\usepackage{yhmath}
%\usepackage[spanish,es-lcroman, es-tabla, es-noshorthands]{babel}

\usepackage{verbatim}

\usepackage[margin=1in]{geometry}
\usepackage{amsmath,amssymb}
\usepackage{multicol}

\usepackage{graphicx}
\graphicspath{{../img/}} 

\newcommand{\class}{Matemáticas 4º Aplicadas}
\newcommand{\examdate}{\today}
\newcommand{\examnum}{Problemas con porcentajes}
\newcommand{\tipo}{D}


\newcommand{\timelimit}{50 minutos}

\renewcommand{\solutiontitle}{\noindent\textbf{Solución:}\enspace}

\pagestyle{head}
\firstpageheader{\includegraphics[width=0.2\columnwidth]{header_left}}{\textbf{Departamento de Matemáticas\linebreak \class}\linebreak \examnum}{\includegraphics[width=0.1\columnwidth]{header_right}}
\runningheader{\class}{\examnum}{Página \thepage\ of \numpages}
\runningheadrule

\pointsinrightmargin % Para poner las puntuaciones a la derecha. Se puede cambiar. Si se comenta, sale a la izquierda.
\extrawidth{-2.4cm} %Un poquito más de margen por si ponemos textos largos.
\marginpointname{ \emph{\points}}


\begin{document}

\noindent
\begin{tabular*}{\textwidth}{l @{\extracolsep{\fill}} r @{\extracolsep{6pt}} }
\textbf{Nombre:} \makebox[3.5in]{\hrulefill} & \textbf{Fecha:}\makebox[1in]{\hrulefill} \\
 & \\
\textbf{Tiempo: \timelimit} & Tipo: \tipo 
\end{tabular*}
\rule[2ex]{\textwidth}{2pt}
Esta prueba tiene \numquestions\ ejercicios. La puntuación máxima es de \numpoints. 
La nota final de la prueba será la parte proporcional de la puntuación obtenida sobre la puntuación máxima. 
\begin{center}


\addpoints
 %\gradetable[h][questions]
	\pointtable[h][questions]
\end{center}

\noindent
\rule[2ex]{\textwidth}{2pt}

\begin{questions}

\question [2] De los 660 alumnos de un centro escolar, 297 hacen deporte regularmente. ¿Qué tanto por ciento no hace deporte?

\begin{solution}$55$\end{solution}

\addpoints


\question [2] 
El precio actual de una vivienda en cierta ciudad es de 212800 \euro. Sabiendo que, en el último año, el precio de la vivienda en ese lugar se ha incrementado en un 12\%, ¿cuánto costaba el año pasado?
\begin{solution}
$190000$
\end{solution}

\question [2] Una empresa de mensajería sabe que seis personas reparten 20 000 paquetes en 8 días, ¿cuántas personas necesitarán para repartir 25 000 paquetes en 10 días?
\begin{solution}
6 personas
\end{solution}

\question [2] El precio de la vivienda ha sufrido dos subidas en los últimos meses: la primera de un 5\% y la segunda de un 12\%. Un piso que costaba inicialmente 120.000 \euro, ¿cuánto cuesta ahora? ¿Cuál ha sido el porcentaje de subida final?
\begin{solution}
$141120.0, 17.60$
\end{solution}


\question [2] Un artículo costaba inicialmente 330 \euro. En enero tuvo una subida de un 8\%; y en febrero bajo un 4\%. ¿Cuál fue el precio final después de estas dos variaciones?
\begin{solution}
$342.144, 3.6799999999999944$
\end{solution}

\question [2] Una cadena de electrodomésticos de dudosa ética sube sus productos un 20\% 15 días antes del Black Friday. Para el Black Friday baja sus precios un 20\% pensando que de esta forma no habrá variación sobre los precios que había antes de la subida. ¿Está en lo cierto? Razona matemáticamente tu respuesta
\begin{solution}
Baja un 4 por ciento 
\end{solution}




\addpoints

\end{questions}

\end{document}
