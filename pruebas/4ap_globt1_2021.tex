\documentclass[addpoints,spanish, 12pt,a4paper]{exam}
\pointpoints{punto}{puntos}
\hpword{Puntos:}
\vpword{Puntos:}
\htword{Total}
\vtword{Total}
\hsword{Resultado:}
\hqword{Ejercicio:}
\vqword{Ejercicio:}

\printanswers

\usepackage[utf8]{inputenc}
\usepackage[spanish]{babel}
\usepackage{eurosym}
\usepackage{yhmath}
%\usepackage[spanish,es-lcroman, es-tabla, es-noshorthands]{babel}


\usepackage[margin=1in]{geometry}
\usepackage{amsmath,amssymb}
\usepackage{multicol}

\usepackage{graphicx}
\graphicspath{{../img/}} 

\newcommand{\class}{Matemáticas 4º Aplicadas}
\newcommand{\examdate}{\today}
\newcommand{\examnum}{Global trimestre 1}
\newcommand{\tipo}{A}


\newcommand{\timelimit}{50 minutos}

\renewcommand{\solutiontitle}{\noindent\textbf{Solución:}\enspace}

\pagestyle{head}
\firstpageheader{\includegraphics[width=0.2\columnwidth]{header_left}}{\textbf{Departamento de Matemáticas\linebreak \class}\linebreak \examnum}{\includegraphics[width=0.1\columnwidth]{header_right}}
\runningheader{\class}{\examnum}{Página \thepage\ of \numpages}
\runningheadrule


\begin{document}

\noindent
\begin{tabular*}{\textwidth}{l @{\extracolsep{\fill}} r @{\extracolsep{6pt}} }
\textbf{Nombre:} \makebox[3.5in]{\hrulefill} & \textbf{Fecha:}\makebox[1in]{\hrulefill} \\
 & \\
\textbf{Tiempo: \timelimit} & Tipo: \tipo 
\end{tabular*}
\rule[2ex]{\textwidth}{2pt}
Esta prueba tiene \numquestions\ ejercicios. La puntuación máxima es de \numpoints. 
La nota final de la prueba será la parte proporcional de la puntuación obtenida sobre la puntuación máxima. %Para la evaluación de pendientes de 3ºESO o 2ºPMAR se tendrán en cuenta los apartados 1,2,3 y 6: 

\begin{center}


\addpoints
 %\gradetable[h][questions]
	\pointtable[h][questions]
\end{center}

\noindent
\rule[2ex]{\textwidth}{2pt}

\begin{questions}

\question[3] Efectúa y simplifica:
\noaddpoints % to omit double points count
\begin{parts}
\part[1] $\dfrac{3}{2}-\dfrac{4}{5}:\dfrac{1}{2}+\dfrac{3}{4}\cdot\dfrac{1}{3}$ 
\begin{solution}$= \dfrac{3}{2}-\dfrac{8}{5}+\dfrac{1}{4}=\dfrac{30}{20}-\dfrac{32}{20}+\dfrac{5}{20}=\dfrac{3}{20}$ \end{solution}
\part[1] $\dfrac{1}{6}-\dfrac{5}{3}\left(\dfrac{4}{5}-\dfrac{1}{3}\right)-\dfrac{1}{2}:\dfrac{3}{4}$
\begin{solution}$=\dfrac{1}{6}-\dfrac{5}{3}\cdot\dfrac{7}{15}-\dfrac{2}{3}=\dfrac{1}{6}-\dfrac{7}{9}-\dfrac{2}{3}=\dfrac{3}{18}-\dfrac{14}{18}-\dfrac{12}{18}=-\dfrac{23}{18}$ \end{solution}
% \part[1] $\left(\dfrac{3}{5}-\dfrac{2}{3}\right)^3\cdot\left[\left(\dfrac{1}{2}+\dfrac{1}{3}\right)^3:\left(\dfrac{5}{3}-1\right)^3\right]$
% \begin{solution}$=\left(\dfrac{9-10}{15}\right)^3\cdot\left[\left(\dfrac{3+2}{6}\right)^3:\left(\dfrac{2}{3}\right)^3\right]=$ \\ 
% $=\left(-\dfrac{1}{15}\right)^3\cdot\left[\left(\dfrac{5}{6}\right)^3:\left(\dfrac{2}{3}\right)^3\right]=\left(-\dfrac{1}{15}\right)^3\cdot\left(\dfrac{5}{4}\right)^3=\left(-\dfrac{1}{12}\right)^3=-\dfrac{1}{1728}$ \end{solution}
\end{parts}
\addpoints

\question[2] Simplifica utilizando las propiedades de las potencias:
\noaddpoints % to omit double points count
\begin{parts}
\part[1] $\dfrac{3^4\cdot 3 \cdot 9^2}{3^0\cdot 3 \cdot 27}$ 
\begin{solution}$=\dfrac{3^5\cdot  \left(3^2\right)^2}{3 \cdot \left(3^3\right)}=\dfrac{3^5\cdot 3^4}{3^4}=3^5=243$ \end{solution}
% \part[1] $\dfrac{6^3\cdot 3^{-2}}{3^6\cdot 2^{-2}}$
% \begin{solution}$=\dfrac{\left(2\cdot3\right)^3\cdot 2^{2}}{3^6\cdot 3^{2}}=\dfrac{\left(2^3\cdot3^3\right)\cdot 2^{2}}{3^6\cdot 3^{2}}=\dfrac{2^5\cdot3^3}{3^8}=\left(\dfrac{2}{3}\right)^5$ \end{solution}
\end{parts}
\addpoints

% \question Responde a las siguientes cuestiones:
% \begin{parts}
% \part[1] Pasa a notación decimal los siguientes números:
% \begin{itemize}
% \item $ \frac{5}{2}$
% \item $ \frac{4}{3}$
% \end{itemize}
% \begin{solution}\begin{itemize}
% \item $ 2,5$
% \item $ 1,\wideparen{3}$
% \end{itemize} \end{solution}
% \part Pasa a fracción irreducible los siguientes números:
% \begin{itemize}
% \item $ 7,5$
% \item $ 6$
% \item $ \sqrt{6}$
% \end{itemize}
% \begin{solution}\begin{itemize}
% \item $ \frac{75}{10}=\frac{15}{2}$
% \item $ 1,\wideparen{3}$
% \item No se puede, es irracional
% \end{itemize} \end{solution}
% \end{parts}
% \addpoints

\question Juan se gasta 2/3 del dinero en ropa y 1/4 del total en comida.:
\noaddpoints % to omit double points count
\begin{parts}
\part[1] ¿Cuál es la fracción gastada?
\begin{solution}$\frac{2}{3}+\frac{1}{4}=\frac{8+3}{12}=\frac{11}{12}$ \end{solution}
\part[1] ¿Qué fracción le queda por gastar?
\begin{solution}$1-\frac{11}{12}=\frac{1}{12}$ \end{solution}
\part[1] Si salió de casa con 180 \euro, ¿qué cantidad no se ha gastado?
\begin{solution} $\dfrac{1}{12} \ de \ 180=\dfrac{1\cdot180}{12}=15$ \euro \end{solution}
\end{parts}
\addpoints

% \question Responde a las siguientes cuestiones:
% \begin{parts}
% \part[1] Da una aproximación, con tres cifras significativas, para cada una de las
% siguientes cantidades:
% \begin{itemize}
% \item $854 238$ personas
% \item $3,1694$ m
% \item $928 412$ mg
% \end{itemize}
% \begin{solution} 
% \begin{itemize}
% \item 854 miles de personas
% \item 3,17 m
% \item 928 miles de mg
% \end{itemize}
% \end{solution}
% \part[1] ¿Cuáles el error absoluto cometido en cada caso?
% \begin{solution} \begin{itemize}
% \item Error absoluto: 854 238 - 854 000 = 238 personas. Error relativo: $\frac{238}{854238}\approx0,0003 $
% \item Error absoluto: 3,1694 - 3,17 = - 0,0006 m. Error relativo: $\frac{-0.0006}{3,1694}\approx-0,0002 $
% \item Error absoluto: 412 mg. Error relativo: $\frac{412}{928412}\approx0,0004 $
% \end{itemize}
% \end{solution}
% \end{parts}

% \addpoints



\question Responde a las siguientes cuestiones relacionadas con la notación científica:
\begin{parts}
\part[1] Expresa en notación científica cada una de estas cantidades:
\begin{itemize}
\item A = $328 000 000 000$
\item B = $0,000000012$
\end{itemize}
\begin{solution} \begin{itemize}
\item A = $3,28\cdot10^{11}$
\item B = $1,2 \cdot10^{-8}$
\end{itemize}\end{solution}
\part[1] Escribe en forma decimal los siguientes números dados en notación científica:
\begin{itemize}
\item C = $2,25\cdot10^8$
\item D = $3,2\cdot 10^{-4}$
\end{itemize}
\begin{solution} \begin{itemize}
\item C = $225 000 000$
\item D = $0,00032$
\end{itemize}\end{solution}
\part[2] Calcula operando en notación científica: $ \left( A + C \right) \cdot B$
\begin{solution}$\left( A + C \right) \cdot B= (3,28 \cdot 10^{11} + 2,25 \cdot 10^8 ) \cdot (1,2 \cdot 10^{-8} ) = (3,28225 \cdot 10^{11} ) \cdot (1,2 \cdot 10^{-8} ) 
= 3,9387 \cdot 10^3 = 3 938,7$ \end{solution}

\end{parts}

\addpoints

% \question Responde a las siguientes cuestiones relacionadas con esta operación: $\left(5,28\cdot10^4+2,81\cdot10^5\right)^2$
% \begin{parts}
% \part[1] Halla el resultado, con ayuda de la calculadora, dando el resultado en notación científica con tres cifras significativas:
% \begin{solution} $1,11\cdot10^{11}$
% \end{solution}
% \part[2] Da una cota para el error absoluto y otra para el error relativo cometidos al dar el valor aproximado.
% \begin{solution} $E_a<5\cdot10^8$ y $E_a<\dfrac{5\cdot10^8}{1,11\cdot10^{11}}\approx0,0045$
% \end{solution}
% \end{parts}

% \addpoints



% \question La masa de la Luna es $7,35 \cdot 10^{22}$ kg, la de Mercurio $3,302\cdot10^{23}$ kg y la de la
% Tierra es $5,98\cdot10^{24}$ kg.
% \begin{parts}
% \part[2] Calcula las veces que la masa de la Luna es menor que la masa de Mercurio
% \begin{solution} $3,302 \cdot 10^{23} : 7,35 \cdot 10^{22} = 33,02 \cdot 10^{22} : 7,35 \cdot 10^{22} = 4,49$. La masa de la Luna es 4,49 veces menor que la de Mercurio.
% \end{solution}
% \part[2] Halla la diferencia entre las masas de la Tierra y de Mercurio
% \begin{solution} Diferencia de masa Tierra - Mercurio: $ 5,98 \cdot 10^{24} - 3,302 \cdot 10^{23} = 59,8 \cdot 10^{23} - 3,302 \cdot 10^{23} = 56,498 \cdot 10^{23} = 5,6498 \cdot 10^{24}$ kg
% \end{solution}
% \end{parts}

% \addpoints

% \question[2] Indica a cuáles de los conjuntos
% $\mathbb{N}$, $\mathbb{Z}$, $\mathbb{Q}$, $\mathbb{R}$ pertenecen cada uno de los siguientes números:
% \begin{center}
% \begin{tabular}{|c |c |c |c |c|}\hline
% &$\mathbb{N}$& $\mathbb{Z}$& $\mathbb{Q}$&$\mathbb{R}$\\ 
% \hline
% $\frac{3}{4}$&&&&\\
% \hline
% $\sqrt[3]{-27}$&&&&\\
% \hline
% $1,\wideparen{3}$&&&&\\
% \hline
% $-\frac{16}{4}$&&&&\\
% \hline
% $-\sqrt{25}$&&&&\\
% \hline
% $\sqrt{8}$&&&&\\
% \hline
% $4$&&&&\\
% \hline
% $\pi$&&&&\\
% \hline
% $\sqrt{-4}$&&&&\\
% \hline
% $\frac{26}{13}$&&&&\\
% \hline
% \end{tabular}

% \end{center}

% \begin{solution}
% \begin{tabular}{|c |c |c |c |c|}\hline
% &$\mathbb{N}$& $\mathbb{Z}$& $\mathbb{Q}$&$\mathbb{R}$\\ 
% \hline
% $\frac{3}{4}$&&&X&X\\
% \hline
% $\sqrt[3]{-27}$&&X&X&X\\
% \hline
% $1,\wideparen{3}$&&&X&X\\
% \hline
% $-\frac{16}{4}$&&X&X&X\\
% \hline
% $-\sqrt{25}$&&X&X&X\\
% \hline
% $\sqrt{8}$&&&&X\\
% \hline
% $4$&X&X&X&X\\
% \hline
% $\pi$&&&&X\\
% \hline
% $\sqrt{-4}$&&&&\\
% \hline
% $\frac{26}{13}$&X&X&X&X\\
% \hline
% \end{tabular}
% \end{solution}

% \addpoints

% \question[2] Representa en la recta real y en forma de intervalo el siguiente conjunto numérico:
% \addpoints % to omit double points count
% $$\left\{ x \in \mathbb{R} \left| -2 \leqslant x < 4 \right. \right\}$$

% \begin{solution}
% $\left[-2 \ , 4\right)$ 
% \end{solution}

% \question Calcula los siguientes radicales utilizando la definición o la notación en forma de potencia. (Justifica tus respuestas):
% \begin{parts}
% \part[1] $\sqrt[3]{2744}$
% \begin{solution} $14$ porque $14^3=2744$
% \end{solution}
% \part[2] $\sqrt{2}\cdot\sqrt{50}$
% \begin{solution} $=2^\frac{1}{2}\cdot50^\frac{1}{2}=\left(2\cdot50\right)^\frac{1}{2}=100^\frac{1}{2}=\left(10^2\right)^\frac{1}{2}=10 $
% \end{solution}
% \end{parts}

% \addpoints

\question[3] Cinco trabajadores tardan 16 días en construir una pequeña caseta de aperos  trabajando  6  horas  diarias.  ¿Cuántos  trabajadores  serán necesarios para construir dicha casita en 10 días si trabajan 8 horas diarias?
\begin{solution}
6 trabajadores
\end{solution}
\question [3] Un automóvil ha tardado en hacer el recorrido Madrid-Zaragoza tres horas y cuarto a una media de 100 km/h. ¿Cuánto tardará un autobús a una media de 90 km/h?
\begin{solution}
175,5 minutos = 2 h 55 min 30 seg 
\end{solution}

\question [3] Una  piscina  portátil  ha  tardado  en  llenarse  seis  horas  utilizando cuatro  grifos iguales. ¿Cuántos grifos, iguales a los anteriores, serían necesarios para llenarla en 3 horas?
\begin{solution}
8 grifos 
\end{solution}



\end{questions}

\end{document}
