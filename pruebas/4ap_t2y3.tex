\documentclass[addpoints,spanish, 12pt,a4paper]{exam}
\pointpoints{punto}{puntos}
\hpword{Puntos:}
\vpword{Puntos:}
\htword{Total}
\vtword{Total}
\hsword{Resultado:}
\hqword{Ejercicio:}
\vqword{Ejercicio:}

\printanswers

\usepackage[utf8]{inputenc}
\usepackage[spanish]{babel}
\usepackage{eurosym}
%\usepackage[spanish,es-lcroman, es-tabla, es-noshorthands]{babel}


\usepackage[margin=1in]{geometry}
\usepackage{amsmath,amssymb}
\usepackage{multicol}

\usepackage{graphicx}
\graphicspath{{../img/}} 

\newcommand{\class}{Matemáticas 4º Aplicadas}
\newcommand{\examdate}{\today}
\newcommand{\examnum}{Números Reales}
\newcommand{\tipo}{A}


\newcommand{\timelimit}{50 minutos}

\renewcommand{\solutiontitle}{\noindent\textbf{Solución:}\enspace}

\pagestyle{head}
\firstpageheader{\includegraphics[width=0.2\columnwidth]{header_left}}{\textbf{Departamento de Matemáticas\linebreak \class}\linebreak \examnum}{\includegraphics[width=0.1\columnwidth]{header_right}}
\runningheader{\class}{\examnum}{Página \thepage\ of \numpages}
\runningheadrule


\begin{document}

\noindent
\begin{tabular*}{\textwidth}{l @{\extracolsep{\fill}} r @{\extracolsep{6pt}} }
\textbf{Nombre:} \makebox[3.5in]{\hrulefill} & \textbf{Fecha:}\makebox[1in]{\hrulefill} \\
 & \\
\textbf{Tiempo: \timelimit} & Tipo: \tipo 
\end{tabular*}
\rule[2ex]{\textwidth}{2pt}
Esta prueba tiene \numquestions\ ejercicios. La puntuación máxima es de \numpoints. 
La nota final de la prueba será la parte proporcional de la puntuación obtenida sobre la puntuación máxima. Para la evaluación de pendientes de 3ºESO o 2ºPMAR se tendrán en cuenta los apartados ???: 

\begin{center}


\addpoints
 %\gradetable[h][questions]
	\pointtable[h][questions]
\end{center}

\noindent
\rule[2ex]{\textwidth}{2pt}

\begin{questions}

\question La masa de la Luna es $7,35 \cdot 10^{22}$ kg, la de Mercurio $3,302\cdot10^{23}$ kg y la de la
Tierra es $5,98\cdot10^{24}$ kg.
\begin{parts}
\part[2] Calcula las veces que la masa de la Luna es menor que la masa de Mercurio
\begin{solution}
\end{solution}
\part[2] Halla la diferencia entre las masas de la Tierra y de Mercurio
\begin{solution}
\end{solution}
\end{parts}

\question Responde a las siguientes cuestiones relacionadas con esta operación: $\left(5,28\cdot10^4+2,81\cdot10^5\right)^2$
\begin{parts}
\part[2] Halla el resultado, con ayuda de la calculadora, dando el resultado en notación científica con tres cifras significativas:
\begin{solution} $1,11\cdot10^{11}$
\end{solution}
\part[2] Da una cota para el error absoluto y otra para el error relativo cometidos al dar el valor aproximado.
\begin{solution} $E_a<5\cdot10^8$ y $E_a<\dfrac{5\cdot10^8}{1,11\cdot10^{11}}\approx0,0045$
\end{solution}
\end{parts}

\addpoints

\question Responde a las siguientes cuestiones:
\begin{parts}
\part[2] Da una aproximación, con tres cifras significativas, para cada una de las
siguientes cantidades:
\begin{itemize}
\item $854 238$ personas
\item $3,1694$ m
\item $928 412$ mg
\end{itemize}
\begin{solution} 
\begin{itemize}
\item 854 miles de personas
\item 3,17 m
\item 928 miles de mg
\end{itemize}
\end{solution}
\part[2] ¿Cuáles son los errores absoluto y relativo cometidos en cada caso?
\begin{solution} $E_a<5\cdot10^8$ y $E_a<\dfrac{5\cdot10^8}{1,11\cdot10^{11}}\approx0,0045$
\end{solution}
\end{parts}

\addpoints

\question[3] Calcula:
\noaddpoints % to omit double points count
\begin{parts}
\part[1] $ -3 + 7 [-4 - (-12) : (-6) + 4 \cdot (-3)]$ 
\begin{solution}$= -3 + 7 [-4 - 2 - 12] = -3 + 7 \cdot (-18) = -3 - 126 = -129$ \end{solution}
\part[1] $14 : (-2) + (-5) : 5 - (-3) + 12$
\begin{solution}$=-7 - 1 + 3 + 12 = 7$ \end{solution}
\part[1] $7\cdot\left[12 + \left(-6 + 4 + 8\right)\right] - \left(-2\right) \cdot \left[5 - 3 \cdot \left(2 + 3 - 6\right)\right]$
\begin{solution}$= 7 [12 + 6] + 2 \cdot [5 - 3 \cdot (-1)] = 7 \cdot [18] + 2 \cdot [5 + 3] = 126 + 2 \cdot 8 = 126 + 16 = 142$ \end{solution}
\end{parts}
\addpoints

\question[3] Efectúa y simplifica:
\noaddpoints % to omit double points count
\begin{parts}
\part[1] $\dfrac{3}{2}-\dfrac{4}{5}:\dfrac{1}{2}+\dfrac{3}{4}\cdot\dfrac{1}{3}$ 
\begin{solution}$= \dfrac{3}{2}-\dfrac{8}{5}+\dfrac{1}{4}=\dfrac{30}{20}-\dfrac{32}{20}+\dfrac{5}{20}=\dfrac{3}{20}$ \end{solution}
\part[1] $\dfrac{1}{6}-\dfrac{5}{3}\left(\dfrac{4}{5}-\dfrac{1}{3}\right)-\dfrac{1}{2}:\dfrac{3}{4}$
\begin{solution}$=\dfrac{1}{6}-\dfrac{5}{3}\cdot\dfrac{7}{15}-\dfrac{2}{3}=\dfrac{1}{6}-\dfrac{7}{9}-\dfrac{2}{3}=\dfrac{3}{18}-\dfrac{14}{18}-\dfrac{12}{18}=-\dfrac{23}{18}$ \end{solution}
\part[1] $\left(\dfrac{3}{5}-\dfrac{2}{3}\right)^3\cdot\left[\left(\dfrac{1}{2}+\dfrac{1}{3}\right)^3:\left(\dfrac{5}{3}-1\right)^3\right]$
\begin{solution}$=\left(\dfrac{9-10}{15}\right)^3\cdot\left[\left(\dfrac{3+2}{6}\right)^3:\left(\dfrac{2}{3}\right)^3\right]=$ \\ 
$=\left(-\dfrac{1}{15}\right)^3\cdot\left[\left(\dfrac{5}{6}\right)^3:\left(\dfrac{2}{3}\right)^3\right]=\left(-\dfrac{1}{15}\right)^3\cdot\left(\dfrac{5}{4}\right)^3=\left(-\dfrac{1}{12}\right)^3=-\dfrac{1}{1728}$ \end{solution}
\end{parts}
\addpoints

\question[2] Simplifica utilizando las propiedades de las potencias:
\noaddpoints % to omit double points count
\begin{parts}
\part[1] $\dfrac{3^4\cdot 3 \cdot 9^2}{3^0\cdot 3 \cdot 27}$ 
\begin{solution}$=\dfrac{3^5\cdot  \left(3^2\right)^2}{3 \cdot \left(3^3\right)}=\dfrac{3^5\cdot 3^4}{3^4}=3^5=243$ \end{solution}
\part[1] $\dfrac{6^3\cdot 3^{-2}}{3^6\cdot 2^{-2}}$
\begin{solution}$=\dfrac{\left(2\cdot3\right)^3\cdot 2^{2}}{3^6\cdot 3^{2}}=\dfrac{\left(2^3\cdot3^3\right)\cdot 2^{2}}{3^6\cdot 3^{2}}=\dfrac{2^5\cdot3^3}{3^8}=\left(\dfrac{2}{3}\right)^5$ \end{solution}
\end{parts}
\addpoints

\question[3] Juan se gasta 2/3 del dinero en ropa y 1/4 del total en comida.:
\noaddpoints % to omit double points count
\begin{parts}
\part[1] ¿Cuál es la fracción gastada?
\begin{solution}$\frac{2}{3}+\frac{1}{4}=\frac{8+3}{12}=\frac{11}{12}$ \end{solution}
\part[1] ¿Qué fracción le queda por gastar?
\begin{solution}$1-\frac{11}{12}=\frac{1}{12}$ \end{solution}
\part[1] Si salió de casa con 180 \euro, ¿qué cantidad no se ha gastado?
\begin{solution} $\dfrac{1}{12} \ de \ 180=\dfrac{1\cdot180}{12}=15$ \euro \end{solution}
\end{parts}
\addpoints

\question[2] Un jardinero riega en un día 2/5 partes del jardín. ¿Cuántos días tardará en regar todo el jardín? ¿Cuánto ganará si cobra 50 \euro. por día?
\begin{solution} Días tardará: $1:\frac{2}{5}=\frac{5}{2}=2,5$ días  \\
Dinero que ganará: $50\cdot2,5=125$ \euro. \end{solution}


\question[3] Pablo gasta 2/5 del dinero que tenía en comprar fruta. Después, gasta 1/4 de lo que le queda en comprar leche. Sabiendo que le han sobrado 9 \euro. ¿Cuánto dinero tenía al principio?
\begin{solution} Fracción gastada: $\frac{2}{5}+\frac{1}{4}\cdot\frac{3}{5}=\frac{8}{20}+\frac{3}{20}=\frac{11}{20}$  \\
Fracción que le queda: $1-\frac{11}{20}=\frac{9}{20}$ \\
Dinero que tenía: Si $\frac{9}{20}$ del Total $= 9\to$ Total $=\dfrac{9\cdot20}{9}=20$ \euro. \end{solution}

\addpoints


\end{questions}

\end{document}
