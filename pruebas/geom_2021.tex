\documentclass[addpoints,spanish, 12pt,a4paper]{exam}
\pointpoints{punto}{puntos}
\hpword{Puntos:}
\vpword{Puntos:}
\htword{Total}
\vtword{Total}
\hsword{Resultado:}
\hqword{Ejercicio:}
\vqword{Ejercicio:}
\usepackage{pgf,tikz}
\usepackage{tikzit}
%\input{sample.tikzstyles}
\usetikzlibrary{shapes, calc, shapes, arrows, math, babel}

%\printanswers

\usepackage[utf8]{inputenc}
\usepackage[spanish]{babel}
\usepackage{eurosym}
\usepackage{yhmath}
%\usepackage[spanish,es-lcroman, es-tabla, es-noshorthands]{babel}

\usepackage{verbatim}

\usepackage[margin=1in]{geometry}
\usepackage{amsmath,amssymb}
\usepackage{multicol}

\usepackage{graphicx}
\graphicspath{{../img/}} 

\newcommand{\class}{Matemáticas 4º Aplicadas}
\newcommand{\examdate}{\today}
\newcommand{\examnum}{Funciones y Geometría}
\newcommand{\tipo}{A}


\newcommand{\timelimit}{50 minutos}

\renewcommand{\solutiontitle}{\noindent\textbf{Solución:}\enspace}

\pagestyle{head}
\firstpageheader{\includegraphics[width=0.2\columnwidth]{header_left}}{\textbf{Departamento de Matemáticas\linebreak \class}\linebreak \examnum}{\includegraphics[width=0.1\columnwidth]{header_right}}
\runningheader{\class}{\examnum}{Página \thepage\ of \numpages}
\runningheadrule

\pointsinrightmargin % Para poner las puntuaciones a la derecha. Se puede cambiar. Si se comenta, sale a la izquierda.
\extrawidth{-2.4cm} %Un poquito más de margen por si ponemos textos largos.
\marginpointname{ \emph{\points}}


\begin{document}

\noindent
\begin{tabular*}{\textwidth}{l @{\extracolsep{\fill}} r @{\extracolsep{6pt}} }
\textbf{Nombre:} \makebox[3.5in]{\hrulefill} & \textbf{Fecha:}\makebox[1in]{\hrulefill} \\
 & \\
\textbf{Tiempo: \timelimit} & Tipo: \tipo 
\end{tabular*}
\rule[2ex]{\textwidth}{2pt}
Esta prueba tiene \numquestions\ ejercicios. La puntuación máxima es de \numpoints. 
La nota final de la prueba será la parte proporcional de la puntuación obtenida sobre la puntuación máxima. 
%Para la evaluación de pendientes de 3ºESO o 2ºPMAR se tendrán en cuenta los apartados 1.a, 1.c, 1.d, 2.a y 4: 

\begin{center}


\addpoints
 %\gradetable[h][questions]
	\pointtable[h][questions]
\end{center}

\noindent
%\textbf{NOTA:} Los problemas se han de resolver mediante ecuaciones o sistemas. Y los ejercicios mediante métodos diferentes a la resolución por tanteo.
\rule[2ex]{\textwidth}{2pt}

\begin{questions}



\question[1] Hemos salido a medir el edificio. Y hemos obtenido los siguientes datos.\begin{itemize}
\item La sombra del edificio es de 9.23 metros
\item La altura de una persona es 1.70 mts y su sombra es 2.21 mts
\item La altura de otra persona es 1.80 mts y su sombra es 2.34 mts
\end{itemize}
Determina la altura del edifico
\begin{solution} $\dfrac{x}{923}=\dfrac{170}{221} \to x=\dfrac{170\cdot923}{221}= 710 cm$  \end{solution}

\question Tenemos un Tupperware de dimensiones: 20cm de largo, 10cm de ancho y 8cm de alto:
\begin{parts}
\part[1] Si queremos pintarlo por fuera, ¿cuánta superficie hay que pintar?
\begin{solution} (200, 80, 160, 880, 0.08800000000000001) \end{solution}

\part[1] ¿Cuántos litros de sopa cabrán en el tupper sabiendo que un litro es lo mismo que un decímetro cúbico?
\begin{solution} (1600.0, 1.6)  \end{solution}

\part[1] ¿Cuánto pesará el tupper lleno sabiendo que 1 litro de sopa pesa un kilogramo?
\begin{solution} 1.6 kg  \end{solution}

\end{parts}


\question Tenemos una piscina con las siguientes dimensiones:
\begin{center}
\tikzfig{piscina2}
\end{center}
\begin{parts}
\part[2] Si queremos pintarla, ¿cuántos botes de pintura necesitaré si con un bote pinto 100 metros cuadrados de superficie?
\begin{solution}

\end{solution}
\part[1] ¿Cuántos litros necesito para llenarla?
\begin{solution}

\end{solution}
\end{parts}


\question Sabemos que la pirámide de Kefrén tiene 136 mts de altura y el lado de la base mide 215 mts:
\begin{parts}
\part[2] Si queremos pintarla, ¿cuántos botes de pintura necesitaré si con un bote pinto 100 metros cuadrados de superficie?
\begin{solution}
(sqrt(29945),
 173.04623659588788,
 215*sqrt(29945)/2,
 18602.470434057945,
 74409.88173623178)
\end{solution}
\part[1] ¿Cuánto pesará la pirámide si cada metro cúbico de piedra pesa 2500 kg?(Da el resultado en notación científica)
\begin{solution}
(2095533, 2095533.0, 2095533000)
\end{solution}
\end{parts}

\addpoints

\end{questions}

\end{document}
