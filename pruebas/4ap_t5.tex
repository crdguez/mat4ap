\documentclass[addpoints,spanish, 12pt,a4paper]{exam}
\pointpoints{punto}{puntos}
\hpword{Puntos:}
\vpword{Puntos:}
\htword{Total}
\vtword{Total}
\hsword{Resultado:}
\hqword{Ejercicio:}
\vqword{Ejercicio:}

%\printanswers

\usepackage[utf8]{inputenc}
\usepackage[spanish]{babel}
\usepackage{eurosym}
\usepackage{yhmath}
%\usepackage[spanish,es-lcroman, es-tabla, es-noshorthands]{babel}

\usepackage{verbatim}

\usepackage[margin=1in]{geometry}
\usepackage{amsmath,amssymb}
\usepackage{multicol}

\usepackage{graphicx}
\graphicspath{{../img/}} 

\newcommand{\class}{Matemáticas 4º Aplicadas}
\newcommand{\examdate}{\today}
\newcommand{\examnum}{Expresiones algebraicas}
\newcommand{\tipo}{A}


\newcommand{\timelimit}{50 minutos}

\renewcommand{\solutiontitle}{\noindent\textbf{Solución:}\enspace}

\pagestyle{head}
\firstpageheader{\includegraphics[width=0.2\columnwidth]{header_left}}{\textbf{Departamento de Matemáticas\linebreak \class}\linebreak \examnum}{\includegraphics[width=0.1\columnwidth]{header_right}}
\runningheader{\class}{\examnum}{Página \thepage\ of \numpages}
\runningheadrule

\pointsinrightmargin % Para poner las puntuaciones a la derecha. Se puede cambiar. Si se comenta, sale a la izquierda.
\extrawidth{-2.4cm} %Un poquito más de margen por si ponemos textos largos.
\marginpointname{ \emph{\points}}


\begin{document}

\noindent
\begin{tabular*}{\textwidth}{l @{\extracolsep{\fill}} r @{\extracolsep{6pt}} }
\textbf{Nombre:} \makebox[3.5in]{\hrulefill} & \textbf{Fecha:}\makebox[1in]{\hrulefill} \\
 & \\
\textbf{Tiempo: \timelimit} & Tipo: \tipo 
\end{tabular*}
\rule[2ex]{\textwidth}{2pt}
Esta prueba tiene \numquestions\ ejercicios. La puntuación máxima es de \numpoints. 
La nota final de la prueba será la parte proporcional de la puntuación obtenida sobre la puntuación máxima. Para la evaluación de pendientes de 3ºESO o 2ºPMAR se tendrán en cuenta los apartados 1, 2, 3 y 6.a y 6.c: 

\begin{center}


\addpoints
 %\gradetable[h][questions]
	\pointtable[h][questions]
\end{center}

\noindent
\rule[2ex]{\textwidth}{2pt}

\begin{questions}

\question Indica el coeficiente, la parte literal y el grado de los siguientes monoimios.
 
\begin{parts}
\part [1] $A=6x^3$

\begin{solution}$ $\end{solution}
\part [1] $B=-3x$

\begin{solution}$ $\end{solution}
\part [1] $C=4x^3$

\begin{solution}$ $\end{solution}
\end{parts}

\addpoints

\question Dados los monomios $A=6x^3$, $B=-3x$, $C=4x^3$, calcula:
\begin{parts}
\part [1] $(C-A)\cdot B$

\begin{solution}$ $\end{solution}
\part [1] $\dfrac{B\cdot C}{A}$

\begin{solution}15,96/0,28=57\end{solution}

\part [1] $\dfrac{A^2}{2B}$
\begin{solution}15,96/0,28=57\end{solution}

\end{parts}
\addpoints


\question [2] Opera y simplifica la siguiente expresión: $$\left(2x^2-3x+1\right)\left(2x-1\right)-\left(4x^3-8x^2+1\right) $$
\begin{solution}
$5x-2$
\end{solution}

\question [2] Multiplica por 6 esta expresión y simplifica: $$\dfrac{2x^2-1}{2}-\dfrac{x-1}{3}-\dfrac{1-x}{6}$$
\begin{solution}
$ 6x^2-x-2$
\end{solution}

\question  Determina el cociente y el resto de las siguientes divisiones:
\begin{parts}
\part [1] $\left(-3x^4+2x^3-6x^2+x-2\right):\left(x-1\right) $

\begin{solution}$ $\end{solution}
\part [1] $\left(-3x^4+6x^2+x-2\right):\left(x+1\right) $

\begin{solution} \end{solution}

\part [2] $\left(-3x^4+6x^2+x-2\right):\left(x+2\right) $
\begin{solution} \end{solution}

\end{parts}
\addpoints

\question Expresa algebraicamente y simplifica los siguientes enunciados:
\begin{parts}
\part [1] La suma de dos números sabiendo que uno es 14 cm mayor que el otro

\begin{solution}  \end{solution}
\part [1] El aŕea de un rectángulo cuya base mide 2 cm más que la altura

\begin{solution} \end{solution}

\part [2] El área de un rombo sabiendo que la longitud de una diagonal es el doble de la otra 
\begin{solution} \end{solution}

\part [2] El perímetro de un rombo sabiendo que la longitud de una diagonal es el doble de la otra 
\begin{solution} \end{solution}

\end{parts}


\addpoints

\end{questions}

\end{document}
