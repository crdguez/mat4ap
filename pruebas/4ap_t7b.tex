\documentclass[addpoints,spanish, 12pt,a4paper]{exam}
\pointpoints{punto}{puntos}
\hpword{Puntos:}
\vpword{Puntos:}
\htword{Total}
\vtword{Total}
\hsword{Resultado:}
\hqword{Ejercicio:}
\vqword{Ejercicio:}
\usepackage{pgf,tikz}
\usetikzlibrary{shapes, calc, shapes, arrows, math, babel}

%\printanswers

\usepackage[utf8]{inputenc}
\usepackage[spanish]{babel}
\usepackage{eurosym}
\usepackage{yhmath}
%\usepackage[spanish,es-lcroman, es-tabla, es-noshorthands]{babel}

\usepackage{verbatim}

\usepackage[margin=1in]{geometry}
\usepackage{amsmath,amssymb}
\usepackage{multicol}

\usepackage{graphicx}
\graphicspath{{../img/}} 

\newcommand{\class}{Matemáticas 4º Aplicadas}
\newcommand{\examdate}{\today}
\newcommand{\examnum}{Funciones y Geometría}
\newcommand{\tipo}{A}


\newcommand{\timelimit}{50 minutos}

\renewcommand{\solutiontitle}{\noindent\textbf{Solución:}\enspace}

\pagestyle{head}
\firstpageheader{\includegraphics[width=0.2\columnwidth]{header_left}}{\textbf{Departamento de Matemáticas\linebreak \class}\linebreak \examnum}{\includegraphics[width=0.1\columnwidth]{header_right}}
\runningheader{\class}{\examnum}{Página \thepage\ of \numpages}
\runningheadrule

\pointsinrightmargin % Para poner las puntuaciones a la derecha. Se puede cambiar. Si se comenta, sale a la izquierda.
\extrawidth{-2.4cm} %Un poquito más de margen por si ponemos textos largos.
\marginpointname{ \emph{\points}}


\begin{document}

\noindent
\begin{tabular*}{\textwidth}{l @{\extracolsep{\fill}} r @{\extracolsep{6pt}} }
\textbf{Nombre:} \makebox[3.5in]{\hrulefill} & \textbf{Fecha:}\makebox[1in]{\hrulefill} \\
 & \\
\textbf{Tiempo: \timelimit} & Tipo: \tipo 
\end{tabular*}
\rule[2ex]{\textwidth}{2pt}
Esta prueba tiene \numquestions\ ejercicios. La puntuación máxima es de \numpoints. 
La nota final de la prueba será la parte proporcional de la puntuación obtenida sobre la puntuación máxima.

\begin{center}


\addpoints
 %\gradetable[h][questions]
	\pointtable[h][questions]
\end{center}

\noindent

\rule[2ex]{\textwidth}{2pt}

\begin{questions}

\question Una compañía de teléfonos me cobra una cantidad fija al mes: 1 \euro . Además me cobran 50 céntimos por cada hora de llamadas. Queremos reflejar en forma de función la factura mensual (lo que pago al mes)
\begin{parts}
\part[1] ¿Cuáles son la variables dependientes e independientes de la función?
\begin{solution}
$x$ = tiempo, $y$ = dinero
\end{solution}
\part[1] Haz una tabla de valores que refleje dicha variable

\begin{solution}
\begin{center}
\begin{tabular}{|c |c |}\hline
$x$ & $y$\\ 
\hline
$1$&$1.5$\\
\hline
$2$&$2$\\
\hline
$3$&$2.5$\\
\hline
$4$&$3$\\
\hline
\end{tabular}
\end{center}
\end{solution}
\part[1] Representa gráficamente los valores anteriores y únelos para determinar la gráfica de la función\\
\begin{tikzpicture}[line cap=round,line join=round,>=triangle 45,x=1cm,y=1cm, scale=0.6]
\draw [color=lightgray,dash pattern=on 1pt off 1pt, xstep=1cm,ystep=1cm] (-3.6,-3.4) grid (20.1,10.1);
\draw[<->,color=black] (-3.6,0) -- (20.1,0);
\foreach \x in {-3,-2,-1,1,2,3,4,5,6,7,8,9,10,11,12,13,14,15,16,17,18,19,20}
\draw[shift={(\x,0)},color=black] (0pt,1pt) -- (0pt,-1pt) node[below] {\footnotesize $\x$};
\draw[<->,color=black] (0,-3.43158220601634095) -- (0,10.1);
\foreach \y in {-3,-2,-1,1,2,3,4,5,6,7,8,9,10}
\draw[shift={(0,\y)},color=black] (2pt,0pt) -- (-2pt,0pt) node[left] {\footnotesize $\y$};
%\draw[color=black] (0pt,-10pt) node[right] {\footnotesize $0$};
%\clip(-0.6129302567150502,-0.43158220601634095) rectangle (9.010648940148005,7.8783927087822985);
\end{tikzpicture}
\begin{solution}
\begin{tikzpicture}[line cap=round,line join=round,>=triangle 45,x=1cm,y=1cm, scale=0.4]
\draw [color=lightgray,dash pattern=on 1pt off 1pt, xstep=1cm,ystep=1cm] (-3.6,-3.4) grid (20.1,10.1);
\draw[<->,color=black] (-3.6,0) -- (20.1,0);
\foreach \x in {-3,-2,-1,1,2,3,4,5,6,7,8,9,10,11,12,13,14,15,16,17,18,19,20}
\draw[shift={(\x,0)},color=black] (0pt,1pt) -- (0pt,-1pt) node[below] {\footnotesize $\x$};
\draw[<->,color=black] (0,-3.43158220601634095) -- (0,10.1);
\foreach \y in {-3,-2,-1,1,2,3,4,5,6,7,8,9,10}
\draw[shift={(0,\y)},color=black] (2pt,0pt) -- (-2pt,0pt) node[left] {\footnotesize $\y$};
%\draw[color=black] (0pt,-10pt) node[right] {\footnotesize $0$};
%\clip(-0.6129302567150502,-0.43158220601634095) rectangle (9.010648940148005,7.8783927087822985);
\draw[->, color=red, domain=0  :16 + 0.1]    plot (\x,{1/2*(\x) + 1}) node[right] {};
\end{tikzpicture}

\end{solution}
\part[1] Da la expresión analítica (o algebraica) de la función
\begin{solution} $y=0.5x+1$
\end{solution}
\part[1] A partir de la expresión analítica, calcula cuánto me facturarán si un mes hablo 200 horas
\begin{solution}
$y=0.5\cdot200+1=101$\euro
\end{solution} 
%\part[1] Determina el dominio y el recorrido de la función
\end{parts}

\question[1] Hemos salido a medir el edificio. Y hemos obtenido los siguientes datos.\begin{itemize}
\item La sombra del edificio es de 9.23 metros
\item La altura de una persona es 1.70 mts y su sombra es 2.21 mts
\item La altura de otra persona es 1.80 mts y su sombra es 2.34 mts
\end{itemize}
Determina la altura del edifico
\begin{solution} $\dfrac{x}{923}=\dfrac{170}{221} \to x=\dfrac{170\cdot923}{221}= 710 cm$  \end{solution}

\question Tenemos un Tupperware de dimensiones: 20cm de largo, 10cm de ancho y 8cm de alto:
\begin{parts}
\part[1] Si queremos pintarlo, ¿cuánta pintura necesitaré si con un bote pinto un metro cuadrado de superficie?
\begin{solution} (200, 80, 160, 880, 0.08800000000000001) \end{solution}

\part[1] ¿Cuántos litros de sopa cabrán en el tupper sabiendo que un litro es lo mismo que un decímetro cúbico?
\begin{solution} (1600.0, 1.6)  \end{solution}

\part[1] ¿Cuánto pesará el tupper lleno sabiendo que 1 litro de sopa pesa un kilogramo?
\begin{solution} 1.6 kg  \end{solution}

\end{parts}

%\question Sabemos que la pirámide de Kefrén tiene 136 mts de altura y el lado de la base mide 215 mts:
%\begin{parts}
%\part[2] Si queremos pintarlo, ¿cuánta pintura necesitaré si con un bote pinto un metro cuadrado de superficie?
%\begin{solution}
%(sqrt(29945),
% 173.04623659588788,
% 215*sqrt(29945)/2,
% 18602.470434057945,
% 74409.88173623178)
%\end{solution}
%\part[1] ¿Cuántos litros cabrían en la pirámide si fuera hueca sabiendo que un litro es lo mismo que un decímetro cúbico?
%\begin{solution}
%(2095533, 2095533.0, 2095533000)
%\end{solution}
%\end{parts}


\addpoints

\end{questions}

\end{document}
